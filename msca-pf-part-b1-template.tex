\documentclass[12pt,draftproposal]{msca-pf}
% {{{ packages

\usepackage[left=15mm,right=15mm,bottom=20mm,top=20mm]{geometry}
\usepackage{mathtools}
\usepackage{graphicx}
\usepackage{mathpartir}

% better environments
\usepackage[nameinlink, noabbrev]{cleveref}
\usepackage[shortlabels]{enumitem}
\usepackage{booktabs}
\usepackage{caption}
\usepackage{subcaption}

% better typography
\usepackage[activate={true,nocompatibility}, % activate protrusion and font expansion
            final,              % enable microtype, use draft to disable
            tracking=true,
            factor=1100,        % more protrusion
            stretch=10,         % smaller values (default 20, 20) to avoid blurring
            shrink=10]{microtype}
\SetTracking{encoding={*}, shape=sc}{40}

% bibliography
\usepackage[style=mla,backend=biber]{biblatex}

% }}}

% {{{ formatting

% NOTE: this needs to be kept so that the sections match the template!
\setcounter{section}{0}

% }}}

% {{{ commands

% add your own fancy commands
% \NewDocumentCommand \mycmd { m } {\textbf{#1}}

\graphicspath{{./graphics}} % Paths whereimages are looked for
\newcommand{\includepdf}[2]{\vcenter{\hbox{\includegraphics[width=#1]{#2.pdf}}}}

\newcommand{\fullwidthbox}[1]{%
    \noindent\makebox[\textwidth]{%
        \fbox{%
            \begin{minipage}{\dimexpr\textwidth-2\fboxsep-2\fboxrule\relax}
                \centering
                #1
            \end{minipage}%
        }%
    }%
}

\newcommand{\proj}{\small\textsc{ScrollNets}}
\newcommand{\WP}[1]{$\mathsf{WP#1}$}
\newcommand{\xstep}[1]{\xrightarrow{#1}}

% }}}

% {{{ title

\title{Part B-1}
\author{Pablo Donato}
\date{\today}

% NOTE: update these as necessary
\mscaidentifier{HORIZON-MSCA-2025-PF-01}
\mscaproject{\textsc{ScrollNets}}

% }}}

% {{{ bibliography

% \usepackage{filecontents}% to embed the file `myreferences.bib` in your `.tex` file

% \begin{filecontents}{myreferences.bib}
% @online{foo12,
%   year = {2012},
%   title = {footnote-reference-using-european-system},
%   url = {http://tex.stackexchange.com/questions/69716/footnote-reference-using-european-system},
% }
% \end{filecontents}

% \addbibresource{MSCA.bib}
\bibliography{MSCA.bib}

% }}}

\begin{document}

\maketitle

% \textit{The text in each section is provided only as guidance and should be deleted
% in the final application (including this text). The tags in the section titles
% and at the end of certain sections should be \textbf{kept as is}. They are used for
% automatic document processing and are highly recommended by the programme.}

% \textit{When in doubt, always consult the official template and programme guide!}

\section{Excellence \mscatag{REL-EVA-RE}}
\label{sc:excellence}

\subsection{Quality and pertinence of the project's research and innovation objectives
    (and the extent to which they are ambitious, and go beyond the state of the art)}
\label{ssc:excellence:quality}

\textbf{Proof assistants} --- also called \emph{interactive theorem provers} (ITPs) --- are software
tools used to rigorously verify formal modelling and reasoning. Contemporary systems such as
\emph{Rocq}, \emph{Lean}, and
\emph{Isabelle}\footcite{the_rocq_development_team_2025_15149629,10.1007/978-3-030-79876-5_37,nipkow2002isabelle}
offer powerful frameworks for constructing formal specifications and proofs: they have been used
successfully in various applications, ranging from the verification of advanced theorems in
mathematics to the certification of complex software artifacts such as programming language
compilers, operating system kernels and cryptographic
protocols\footcite{gonthierFormalProofFour2008,leroyFormalVerificationRealistic2009,kleinSeL4FormalVerification2009,Barthe2014}.
Yet they remain notoriously difficult to learn and use, limiting broader adoption in various
settings where they could bring transformative societal impact, such as:
\begin{itemize}
    \item \textbf{mathematics education}, where they could act as a unifying medium for interactive
    exploration and understanding of mathematical concepts, fostering closer collaboration amongst
    students and offloading some teaching burden (e.g. grading) through
    automation\footcite{minhResearchReportProof2024};
    \item \textbf{formal verification}, where they could bring higher standards of quality assurance
    (QA) to businesses that rely on complex hardware and software systems, especially in
    safety-critical industries such as healthcare, transportation and energy;
    \item \textbf{artificial intelligence} (AI), where the uncertainty inherent to current
    technologies based on probabilistic techniques such as large language models (LLMs) could be
    mitigated by the exact logical reasoning capabilities of ITPs, an approach sometimes termed
    \emph{neurosymbolic AI}.
\end{itemize}
In view of this large potential for applications, it is natural to ask what exactly limits adoption
of the current generation of ITPs. The recent surge of interest arising both in academia and
industry --- in great part due to the popularity of the Lean language and its Mathlib library ---
suggests that social factors such as public communication, community building, and vast amounts of
expository content and learning resources all play an important role in the widespread appropriation
of this complex technology. Even more recently, the promise of a new kind of generative AI free from
so-called ``hallucinations'' that could aid in accelerating scientific discoveries has been a
powerful narrative attracting much attention and funding\footcite{metzMathPathChatbots2024}.

However, many researchers in the field agree that current ITPs suffer from more \emph{foundational}
issues that affect directly their accessibility and ease of use, as well as their ability to scale
to larger developments. While these issues are quite diverse in nature, a recurring theme since the
birth of the technology in the 60s is the overwhelming \textbf{bureaucracy} involved in
formalization efforts: formal proofs require a level of care and detail that is far superior and
much more time consuming than what is expected from informal paper proofs. This has been measured by
the so-called \emph{DeBruijn factor} comparing the size of formal and informal proofs, often
reaching a value of 4 on average\footcite{debruijn_factor}.

The main approach to tame this complexity has been to \emph{automate} the various processes involved
in formalization, which fall roughly within two categories: \emph{elaboration}, concerned with the
translation of requirements expressed in natural language or mathematical notations into precise
logical specifications; and \emph{synthesis}, where the system attempts to automatically generate
(parts of) proofs and programs meeting these specifications. Progress on both fronts is currently
being made with the help of state-of-the-art machine learning techniques, including
LLMs\footcite{blaauwbroekLearningGuidedAutomated2024}. More theoretical research has also been
pursued in \textbf{type theory}, the field studying the logical formalisms at the foundation of all
modern ITPs. They are the ultimate backbone on which relies our \emph{trust} in the output of these
systems, and thus a key differentiator with respect to purely probabilistic approaches to
(generative) AI.
% In particular, the influential program of \emph{homotopy type theory} kickstarted by Fields medalist
% Vladimir Voevodsky has exhibited new ways to understand and mechanize the concept of
% \emph{equality}, which is known as an important weakness of traditional type theories.
% Type-theoretical foundations are the ultimate backbone on which relies our \emph{trust} in the
% output of ITPs, and thus a key differentiator with respect to purely probabilistic approaches to
% (generative) AI.

However, little attention has been devoted by the ITP community to another discipline closely
related to type theory: \textbf{proof theory}. In particular, \emph{structural} proof theory is its
branch concerned with the study of combinatorial structures for representing and manipulating
proofs. One can identify mainly three motivations for this study:
\begin{itemize}
    \item \textbf{Challenge 1} (C1) is the fundamental problem of \textbf{proof identity}, also
    known as Hilbert's 24\textsuperscript{th} problem\footcite{strasburger-problem-2019}. It aims to
    answer the philosophical question ``what is a proof?'', and the mathematical question ``when are
    two proofs equal?''. It is thus intimately related to \emph{homotopy type theory} which also
    investigates the structure of \emph{equality}, a known weakness of many type theories.

    \item \textbf{Challenge 2} (C2) is to find proof systems for \textbf{non-standard logics} ---
    such as \emph{modal}, \emph{intermediate}, \emph{substructural} and \emph{fixpoint} logics ---
    satisfying good enough properties as to render an algorithmic treatment of these logics
    tractable. The most important property in this respect is that of \textbf{cut elimination},
    which is essential both to reduce the complexity of \emph{proof search} (proof \emph{synthesis}
    in ITP terminology), and to ensure productivity of \emph{program execution} through the
    \textbf{Curry-Howard correspondence} (CHC) between proofs and programs. The CHC is also at the
    heart of the \emph{calculus of inductive constructions} (CoIC), which is the type theory used by
    the two leading proof assistants Rocq and Lean. Researchers are increasingly interested in
    type-theoretic formulations of these logics as they provide expressive languages for specifying
    behaviors of programs that go beyond pure functions, including effects (modal logics),
    resource-sensitivity (modal and substructural logics) and recursion (modal and fixpoint
    logics)\footcite{tangModalEffectTypes2025,marshallLinearityUniquenessEntente2022,cloustonGuardedLambdaCalculusProgramming2017}.
    
    \item \textbf{Challenge 3} (C3) is to improve the \textbf{efficiency} of computational
    procedures on proofs, but also on programs through the CHC. A well-established principle in
    computer science and software engineering is that choosing appropriate data structures for a
    problem can lead to orders-of-magnitude improvements in efficiency. Finding the right data
    structures for such general classes of objects as proofs and programs is thus a very enticing
    goal with wide implications, including faster automation in ITPs.
\end{itemize}

In the past decades, two families of proof formalisms have emerged to tackle these challenges:
\begin{itemize}
    \item \textbf{Graphical proof systems} represent proof objects as \emph{graphs} instead of
    \emph{trees} of inference rules.
    % Many programming systems incorporate graph representations of programs in their compilation
    % pipeline in order to optimize the flow of computation, including popular machine learning
    % frameworks based on neural networks such as TensorFlow and PyTorch. Indeed, graphs appear
    % naturally when one wants to abstract from the order in which operations are sequenced, to
    % leverage \emph{parallelism} or to maximize \emph{sharing} of computations.
    \emph{Proof nets}\footcite{girard-linear-1987} are one of the first graphical proof formalisms,
    meant to reduce the bureauceacy of proofs in linear logic (C2) in order to identify their
    essence (C1). Further developments stemming from proof nets like the \emph{geometry of
    interaction} and \emph{interaction nets} have found applications in program optimization for
    hardware synthesis and parallelized
    execution\footcite{ghicaGeometrySynthesisStructured2007,mackieInteractionNetImplementation2011}
    (C3). The \emph{combinatorial proofs} of Hughes are direct decendants of proof nets mainly
    concerned with C1\footcite{Hughes_2006}, and \emph{string diagrams} from category theory have
    been applied to logic and programming languages, with well-known connections to proof
    nets\footcite{piedeleuIntroductionStringDiagrams2025}.

    \item \textbf{Deep inference} generalizes Gentzen-style proof systems by allowing inference
    rules to be applied at any depth inside of a formula, rather than being restricted to its
    top-level logical connective. The terminology of ``deep inference'' was proposed by Alessio
    Guglielmi, who invented the \emph{calculus of structures} to overcome the inability of sequent
    calculus to capture the substructural logic $\mathsf{MV}$\footcite{Guglielmi1999ACO}. Since
    then, calculi of structures and so-called \emph{nested sequent calculi} have been introduced to
    give proof systems enjoying cut-elimination to most substructural, modal and intermediate
    logics\footcite{kuznets_maehara-style_2019,postniece_proof_2010} (C2). Deep inference has also
    been used in the study of proof complexity, providing in some cases exponential speedup over
    sequent calculus with respect to proof size, as well as quasipolynomial-time
    cut-elimination\footcite{dasRelativeProofComplexity2015a,bruscoliQuasipolynomialNormalisationDeep2016a}
    (C3). Lastly, many deep inference formalisms enjoy CHC-style interpretations with variants of
    $\lambda$-calculus, which also improve space
    efficiency\footcite{guenot_nested_2013,gundersenAtomicLambdaCalculus2013}.
\end{itemize}

The experienced researcher (ER) has accumulated significant knowledge of graphical and deep
inference proof systems, as well as their applications to ITPs. This expertise was developed during
his PhD thesis\footcite{donatoDeepInferenceGraphical2024} titled ``Deep Inference for Graphical
Theorem Proving'', where he designed various proof formalisms that enable a novel approach to
interactive theorem proving based on \textbf{direct manipulation} of logical statements in a
graphical user interface (GUI). This extends earlier works on \emph{Proof-by-Pointing}\footcite{PbP}
and \emph{Proof-by-Linking}\footcite{Chaudhuri2013} where proofs are constructed through
\emph{click} and \emph{drag-and-drop} gestures on formulas, to a more encompassing paradigm termed
\emph{Proof-by-Action} (PbA). The goal is to improve accessibility and usability of ITPs by focusing
on better principles for \emph{human interaction}, complementing more mainstream research around
machine automation. Applying a mixture of graphical and deep inference proof theory to that effect
is a highly original endeavor, with no similar efforts in the contemporary research landscape.

\begin{figure}
  \captionsetup[subfigure]{justification=centering}
  \centering
  \begin{subfigure}[b]{0.64\textwidth}
    \begin{mathpar}
      \includepdf{2cm}{modusponens1}
      ~~~\xstep{\mathsf{Deit}}~~~
      \includepdf{2cm}{modusponens2}
      ~~~\xstep{\mathsf{Close}}~~~
      \includepdf{1.25cm}{modusponens3}
      ~~~\xstep{\mathsf{Del}}~~~
      \includepdf{0.25cm}{modusponens4}
    \end{mathpar}
    \caption{Dynamic rewriting from premiss $a \wedge (a \Rightarrow b)$ to conclusion $b$}
    \label{fig:modus-ponens-dynamic}
  \end{subfigure}
  \begin{subfigure}[b]{0.35\textwidth}
    \centering
    $\includepdf{2.25cm}{modusponens-scrollnet}$
    \caption{Static representation of rules as arrows in a bigraph}
    \label{fig:modus-ponens-static}
  \end{subfigure}
  \caption{Proof of \textit{modus ponens} $a \wedge (a \Rightarrow b) \vdash b$ in scroll nets}
  \label{fig:modus-ponens}
\end{figure}

In continuation of this programme, the ER has introduced in his last
preprint\footcite{donatoScrollNets2025} a new graphical framework called \textbf{scroll nets}. It is
based on a long forgotten diagrammatic proof system called \emph{existential graphs} (EGs), invented
by the famous philosopher and logician Charles Sanders Peirce at the dusk of the
19\textsuperscript{th} century --- thus predating the very existence of proof theory and computer
science. Proofs in EGs are defined by a small set of inference rules that \emph{dynamically} rewrite
diagrams in contexts of arbitrary depth (Figure~\ref{fig:modus-ponens-dynamic}), thus combining
features of both deep inference and string diagrams\footcite{bonchi_diagrammatic_2024}. Scroll nets
provide a \emph{static} way to represent proofs in EGs by recording explicitly applications of
inference rules in a \emph{directed graph} (Figure~\ref{fig:modus-ponens-static}), similarly to
proof nets and combinatorial proofs. Crucially, this graph shares the same nodes as the
(tree-shaped) statements involved in the proof, making scroll nets a more compact representation
than other graphical proof structures, but also surprisingly a variant of the notion of
\emph{bigraph}. Bigraphs were introduced by Milner as a foundational combinatorial structure
encompassing most models of concurrent/parallel computation, including Petri nets and his own CCS
and $\pi$-calculus\footcite{milnerBigraphicalReactiveSystems2001}. In his preprint, the ER shows how
scroll nets naturally subsume the simply-typed $\lambda$-calculus --- the common kernel of all type
theories used in ITPs --- by identifying a generalization of the notion of \emph{detour} from
natural deduction that abstracts away the shape of formulas.

All these discoveries hint toward the potential of scroll nets as a very expressive framework for
both proof theory and type theory, unifying features found in most of the formalisms that have
emerged from the two disciplines in a natural and efficient way. Moreover, the ER's vision is to
exploit the diagrammatic nature of scroll nets to redesign not only the \emph{backend} but also the
\emph{frontend} of ITPs, making them the \emph{interaction
substrate}\footcite{mackayInteractionSubstratesCombining2025} of a new kind of GUI in the PbA
paradigm.

Given the early stage of the theory of scroll nets and the expertise of the Theory of Computation
group at Birmingham hosting the ER, the {\proj} project will progress towards this vision by
focusing on the following key objectives:

% One goal is to make automation (in particular synthesis) easier to understand
% for humans, so as to tighten the interaction loop and thus enhance the collaboration between the
% user and the system. Having more readable inputs and outputs is also crucial to enable \emph(audits)
% of formal developments by external authorities, for instance in the certification of safety-critical
% software by governmental agencies.

\setlength{\fboxsep}{5pt}
\fullwidthbox{
    \begin{enumerate}
        \item[(K1)] extend the theory of scroll nets to account for \textbf{richer logics}, beyond minimal
        implicative logic;
        \item[(K2)] find natural and efficient \textbf{translations} of state-of-the-art proof
        systems and typed programming models into scroll nets.
    \end{enumerate}
}

This should support the following very ambitious long-term goal of the project:

\setlength{\fboxsep}{5pt}
\fullwidthbox{\textit{ 
    Establish \textbf{scroll nets} as the foundation for a new generation of ITPs
    that support \textbf{interactive refinement} of formal specifications, proofs and programs through
    \textbf{diagrammatic manipulations}.
}}

% At a minimum, address the following aspects:

% \begin{itemize}
%     \item Describe the quality and pertinence of the R\&I objectives; are the
%     objectives measurable and verifiable? Are they realistically achievable?

%     \item Describe how your project goes beyond the state-of-the-art, and the
%     extent to which the proposed work is ambitious.
% \end{itemize}

\subsection{Soundness of the proposed methodology
    (including interdisciplinary approaches, consideration of the gender
    dimension and other diversity aspects if relevant for the research project,
    and the quality of open science practices)}
\label{ssc:excellence:methodology}



\begin{itemize}
    \item \WP{1} Study of sequentialization and normalization in implicative scroll nets
    \item \WP{2} Horizontal and vertical generalization to account for disjunction/sum types, bi-intuitionistic and classical logic
    \item The Functional Machine Calculus (FMC) and Relational Machine Calculus (RMC) are two
    foundational models of computation introduced very recently by W. Heijltjes, also co-inventor of
    intuitionistic combinatorial
    proofs\footcite{heijltjesFunctionalMachineCalculus2023,barrettRelationalMachineCalculus2024,heijltjes_intuitionistic_2019}.
    They achieve respectively a unification of the two main paradigms of declarative programming,
    \emph{functional} and \emph{logic} programming, with the more mainstream paradigm of
    \emph{imperative} programming. An open problem is to find a suitable combination of the FMC and
    RMC that could subsume all three paradigms. \WP{3} will explore the potential of scroll nets as
    a solution in the well-typed fragment, by devising translations of the FMC and RMC that preserve
    both its denotational and operational semantics. This will build on \WP{1} for the operational
    semantics given by detour elimination, and \WP{2} to account for non-determinism with
    $(n,1)$-ary scrolls for sum and empty types.
    \item \WP{4} Recursive generalization to account for (co)inductive types (cyclic proofs)
    % \item \WP{3} Vertical generalization to account for classical and intermediate logics
    \item \WP{5} Study of \textsf{Beta} and \textsf{Gamma} to account for modal logic, higher-order logic and dependent types
\end{itemize}

% At a minimum, address the following aspects:

% \begin{itemize}
%     \item \emph{Overall methodology}: Describe and explain the overall methodology,
%     including the concepts, models and assumptions that underpin your work.
%     Explain how this will enable you to deliver your project’s objectives. Refer
%     to any important challenges you may have identified in the chosen methodology
%     and how you intend to overcome them.

%     \item \emph{Integration of methods and disciplines to pursue the objectives}:
%     Explain how expertise and methods from different disciplines will be brought
%     together and integrated in pursuit of your objectives. If you consider that
%     an inter-disciplinary\footnotemark{} approach is unnecessary in the
%     context of the proposed work, please provide a justification.

%     \footnotetext{\emph{Interdisciplinarity} means the integration of information,
%     data, techniques, tools, perspectives, concepts or theories from two or
%     more scientific disciplines.}

%     \item \emph{Gender dimension and other diversity aspects}: Describe how the
%     gender dimension and other diversity aspects are taken into account in the
%     project's research and innovation content. If you do not consider such a
%     gender dimension to be relevant in your project, please provide a
%     justification.
%     \begin{itemize}
%         \item Remember that that this question relates to the \emph{content} of the
%         planned research and innovation activities, and not to gender balance
%         in the teams in charge of carrying out the project.

%         \item  Sex, gender and diversity analysis refers to biological
%         characteristics and social/cultural factors respectively. For guidance on
%         methods of sex / gender analysis and the issues to be taken into account,
%         please refer to this
%         \href{https://op.europa.eu/en/publication-detail/-/publication/33b4c99f-2e66-11eb-b27b-01aa75ed71a1/language-en}{link}.
%     \end{itemize}

%     \item \emph{Open science practices}: Describe how appropriate open science
%     practices are implemented as an integral part of the proposed methodology.
%     Show how the choice of practices and their implementation is adapted to the
%     nature of your work in a way that will increase the chances of the project
%     delivering on its objectives \emph{[1/2 page]}. If you believe that none of
%     these practices are appropriate for your project, please provide a justification
%     here.

%     \emph{Open science is an approach based on open cooperative work and systematic
%     sharing of knowledge and tools as early and widely as possible in the process.
%     Open science practices include early and open sharing of research (for example
%     through pre-registration, registered reports, pre-prints, or crowd-sourcing);
%     research output management; measures to ensure reproducibility of research
%     outputs; providing open access to research outputs (such as publications,
%     data, software, models, algorithms, and workflows); participation in open
%     peer-review; and involving all relevant knowledge actors including citizens,
%     civil society and end users in the co-creation of R\&I agendas and contents
%     (such as citizen science).}

%     \emph{Please note that this does not refer to outreach actions that may be
%     planned as part of the communication, dissemination and exploitation activities.
%     These aspects should instead be described below under ``Impact''.}

%     \item Proposals selected for funding under Horizon Europe will need to develop
%     a detailed data management plan (DMP) for making their data/research outputs
%     findable, accessible, interoperable and reusable (FAIR) as a deliverable by
%     month 6 and revised towards the end of a project's lifetime. The DMP should
%     describe how research outputs (especially research data) generated and/or
%     collected during the project will be managed so as to ensure that they are
%     findable, accessible, interoperable and reusable.

%     \emph{For guidance on open science practices and research data management,
%     please refer to the relevant section of the
%     \href{https://ec.europa.eu/info/funding-tenders/opportunities/docs/2021-2027/horizon/temp-form/af/af_he-msca-pf_en.pdf}{HE Programme Guide}
%     on the Funding \& Tenders Portal.}
% \end{itemize}

\subsection{Quality of the supervision, training and of the two-way transfer of
    knowledge between the researcher and the host}
\label{ssc:excellence:supervision}

At a minimum, address the following aspects:

\begin{itemize}
    \item Describe the qualifications and experience of the supervisor(s).
    Provide information regarding the supervisors' level of experience on the
    research topic proposed and their track record of work, including main
    international collaborations, as well as the level of experience in
    supervising/training, especially at advanced level (i.e. PhD and postdoctoral
    researchers).

    \item Planned training activities for the researcher (scientific aspects,
    management/organisation, horizontal and key transferrable skills...).

    \item For \emph{European Fellowships}: two-way transfer of knowledge between
    the researcher and host organisation.

    \item For \emph{Global Fellowships}: three-way transfer of knowledge between
    the researcher, host organisation, and associated partner for outgoing phase.

    \item Rationale and added-value of the non-academic placement (if applicable)
    and secondment (if applicable).
\end{itemize}

Employers and/or funders should ensure that a person is clearly identified to
whom researchers can refer for the performance of their professional duties, and
should inform the researchers accordingly.

Such arrangements should clearly define that the proposed supervisors are
sufficiently expert in supervising research, have the time, knowledge, experience,
expertise and commitment to be able to offer the postdoctoral researcher
appropriate support and provide for the necessary progress and review procedures,
as well as the necessary feedback mechanisms.

\textbf{Supervision} is one of the crucial elements of successful research. Guiding,
supporting, directing, advising and mentoring are key factors for a researcher
to pursue his/her career path. In this context, all MSCA-funded projects are
encouraged to follow the recommendations outlined in the
\href{https://data.europa.eu/doi/10.2766/508311}{MSCA Guidelines on Supervision}
\footnotemark{}.

\footnotetext{While the MSCA Guidelines on Supervision are non-binding,
funded-projects are strongly encouraged to take them into account.}

\subsection{Quality and appropriateness of the researcher's professional
    experience, competences and skills}
\label{ssc:excellence:researcher}

Discuss the quality and appropriateness of the researcher’s existing professional
experience in relation to the proposed research project.

\section{Impact \mscatag{IMP-ACT-IA}}
\label{sc:impact}

\subsection{Credibility of the measures to enhance the career perspectives
    and employability of the researcher and contribution to his/her skills
    development}
\label{ssc:impact:career}

At a minimum, address the following aspects:

\begin{itemize}
    \item Specific measures to enhance career perspectives and employability of
    the researcher inside and/or outside academia.
    \item \emph{Expected} contribution of proposed skills development to the
    future career of the researcher.
\end{itemize}

\subsection{Suitability and quality of the measures to maximise expected
    outcomes and impacts, as set out in the dissemination and exploitation plan,
    including communication activities \mscatag{COM-DIS-VIS-CDV}}
\label{ssc:impact:outcomes}

At a minimum, address the following aspects:

\begin{itemize}
    \item \emph{Plan for the dissemination and exploitation activities, including
    communication activities}\footnotemark{}: Describe the planned measures to
    maximize the impact of your project by providing a first version of your ‘plan
    for the dissemination and exploitation including communication activities’.
    Describe the dissemination, exploitation measures that are planned, and the target
    group(s) addressed (e.g. scientific community, end users, financial actors,
    public at large). Regarding communication measures and public engagement
    strategy, the aim is to inform and reach out to society and show the activities
    performed, and the use and the benefits the project will have for citizens.
    Activities must be strategically planned, with clear objectives, start at the
    outset and continue through the lifetime of the project. The description of
    the communication activities needs to state the main messages as well as the
    tools and channels that will be used to reach out to each of the chosen target
    groups.

    \footnotetext{In case your proposal is selected for funding, a more detailed
    Dissemination and Exploitation plan will need to be provided as a mandatory
    project deliverable during project implementation.}

    \item \emph{Strategy for the management of intellectual property, foreseen
    protection measures}: if relevant, discuss the strategy for the management
    of intellectual property, foreseen protection measures, such as patents,
    design rights, copyright, trade secrets, etc., and how these would be used
    to support exploitation.

    \item All measures should be proportionate to the scale of the project, and
    should contain concrete actions to be implemented both during and after the
    end of the project.
\end{itemize}

\subsection{The magnitude and importance of the project’s contribution to the
    expected scientific, societal and economic impacts}
\label{ssc:impact:future}

\begin{itemize}
    \item Provide a narrative explaining how the project’s results are expected
    to make a difference in terms of impact, beyond the immediate scope and
    duration of the project. The narrative should include the components below,
    tailored to your project.

    \item Be specific, referring to the effects of your project, and not R\&I
    in general in this field. State the target groups that would benefit.

    \item The impacts of your project may be:

    \begin{itemize}
        \item \emph{Scientific}: e.g. contributing to specific scientific advances,
        across and within disciplines, creating new knowledge, reinforcing
        scientific equipment and instruments, computing systems (i.e. research
        infrastructures);

        \item \emph{Economic/technological}: e.g. bringing new
        products, services, business processes to the market, increasing efficiency,
        decreasing costs, increasing profits, contributing to standards’ setting,
        etc.

        \item \emph{Societal}: e.g. decreasing CO2 emissions,
        decreasing avoidable mortality, improving policies and decision-making,
        raising consumer awareness.
    \end{itemize}

    \item Only include such outcomes and impacts where your project would make
    a significant and direct contribution. Avoid describing very tenuous links
    to wider impacts.

    \item Give an indication of the magnitude and importance of the project's
    contribution to the expected outcomes and impacts, should the project be
    successful. Provide credible quantified estimates where possible and meaningful.

    ``Magnitude'' refers to how widespread the outcomes and impacts are likely
    to be. For example, in terms of the size of the target group, or the
    proportion of that group, that should benefit over time.

    ``Importance'' refers to the value of those benefits. For example, number of
    additional healthy life years; efficiency savings in energy supply.
\end{itemize}

\mscatagend{COM-DIS-VIS-CDV} \mscatagend{IMP-ACT-IA}

\section{Quality and Efficiency of the Implementation
         \mscatag{QUA-LIT-QL} \mscatag{WRK-PLA-WP} \mscatag{CON-SOR-CS} \mscatag{PRJ-MGT-PM}}
\label{sc:implementation}

\subsection{Quality and effectiveness of the work plan, assessment of risks and
    appropriateness of the effort assigned to work packages}
\label{ssc:implementation:workplan}

At a minimum, address the following aspects:

\begin{itemize}
    \item Brief presentation of the overall structure of the work plan,
    including deliverables and milestones.

    \item Timing of the different work packages and their components;

    \item Mechanisms in place to assess and mitigate risks (of research and/or
    administrative nature).
\end{itemize}

A Gantt chart must be included and should indicate the proposed Work Packages
(WP), major deliverables, milestones, secondments, placements, if applicable.
This Gantt chart counts towards the 10-page limit.

The schedule in the Gantt chart should indicate the number of months elapsed
from the start of the action (Month 1, Month 2, etc.), but no actual dates.

\subsection{Quality and capacity of the host institutions and participating
organisations, including hosting arrangements}
\label{ssc:implementation:host}

At a minimum, address the following aspects:

\begin{itemize}
    \item Hosting arrangements, including integration in the team/institution(s)
    and support services available to the researcher.

    \item Quality and capacity of the participating organisations, including
    infrastructure, logistics and facilities. Additional information should be
    outlined in Part B-2 Section 5 (``Capacity of the Participating Organisations'').
\end{itemize}

Note that for GF, both the quality and capacity of the outgoing Third Country
host and the return host should be outlined.

\subsection*{Associated partners linked to a beneficiary {\protect\footnotemark{}}}
\label{ssc:implementation:partners}

If applicable, outline here the involvement of any 'associated partners linked
to a beneficiary' (in particular, the name of the entity, the type of link with
the beneficiary and the tasks to be carried out).

\footnotetext{See the definitions section of the MSCA Work Programme for
further information.}

\mscatagend{CON-SOR-CS} \mscatagend{PRJ-MGT-PM} \mscatagend{QUA-LIT-QL} \mscatagend{WRK-PLA-WP}
\end{document}

% kate: default-dictionary en_GB;
