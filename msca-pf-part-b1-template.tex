\documentclass[11pt]{msca-pf}
% {{{ packages

\usepackage[left=15mm,right=15mm,bottom=20mm,top=20mm]{geometry}
\usepackage{amsmath}
\usepackage{amssymb}
\usepackage{mathtools}
\usepackage{prftree}
\usepackage{graphicx}
\usepackage{mathpartir}

% better environments
\usepackage[nameinlink, noabbrev]{cleveref}
\usepackage[shortlabels]{enumitem}
\usepackage{tikz}
\usetikzlibrary{shapes.symbols,arrows.meta}
\usepackage{pgfgantt}
\usepackage{booktabs}
\usepackage{caption}
\usepackage{subcaption}

% better typography
\usepackage[activate={true,nocompatibility}, % activate protrusion and font expansion
            final,              % enable microtype, use draft to disable
            tracking=true,
            factor=1100,        % more protrusion
            stretch=10,         % smaller values (default 20, 20) to avoid blurring
            shrink=10]{microtype}
\SetTracking{encoding={*}, shape=sc}{40}

\RedeclareSectionCommand[
    beforeskip=0.5ex,
    afterskip=0.5ex
]{section}
\RedeclareSectionCommand[
    beforeskip=0.3ex,
    afterskip=0.3ex
]{subsection}
\RedeclareSectionCommand[
    beforeskip=0.3ex,
    afterskip=0.3ex
]{subsubsection}

\RedeclareSectionCommand[
  beforeskip=1pt plus 2pt minus 1pt,
  afterskip=-0.5em
]{paragraph}


% bibliography
\usepackage[style=mla,backend=biber]{biblatex}

% }}}

% {{{ formatting

% NOTE: this needs to be kept so that the sections match the template!
\setcounter{section}{0}

% }}}

% {{{ commands

% add your own fancy commands
% \NewDocumentCommand \mycmd { m } {\textbf{#1}}

\graphicspath{{./graphics}} % Paths whereimages are looked for
\newcommand{\includepdf}[2]{\vcenter{\hbox{\includegraphics[width=#1]{#2.pdf}}}}

\newcommand{\fullwidthbox}[1]{%
    \noindent\makebox[\textwidth]{%
        \fbox{%
            \begin{minipage}{\dimexpr\textwidth-2\fboxsep-2\fboxrule\relax}
                \centering
                #1
            \end{minipage}%
        }%
    }%
}

\newcommand{\proj}{\small\textsc{ScrollNets}}
\newcommand{\CH}[1]{$\mathsf{C#1}$}
\newcommand{\KO}[1]{$\mathsf{KO#1}$}
\newcommand{\WP}[1]{$\mathsf{WP#1}$}
\newcommand{\MI}[1]{$\mathsf{M#1}$}
\newcommand{\DE}[1]{$\mathsf{D#1}$}
\newcommand{\TO}[1]{$\mathsf{TO#1}$}
\newcommand{\xstep}[1]{\xrightarrow{#1}}

\newcommand{\TaskCircle}[3]{
    \begin{tikzpicture}
        \fill[fill={rgb,255:red,#1; green,#2; blue,#3}] (0,0) circle (2.5px);
    \end{tikzpicture}
}

% }}}

% {{{ title

\title{ScrollNets: A Diagrammatic Foundation for Next-Generation Interactive Theorem Provers}
\author{Pablo Donato}
\date{\today}

% NOTE: update these as necessary
\mscaidentifier{HORIZON-MSCA-2025-PF-01}
\mscaproject{\textsc{ScrollNets}}

% }}}

% {{{ bibliography

% \usepackage{filecontents}% to embed the file `myreferences.bib` in your `.tex` file

% \begin{filecontents}{myreferences.bib}
% @online{foo12,
%   year = {2012},
%   title = {footnote-reference-using-european-system},
%   url = {http://tex.stackexchange.com/questions/69716/footnote-reference-using-european-system},
% }
% \end{filecontents}

% \addbibresource{MSCA.bib}
\bibliography{MSCA.bib}

% }}}

\begin{document}

\maketitle

% \textit{The text in each section is provided only as guidance and should be deleted
% in the final application (including this text). The tags in the section titles
% and at the end of certain sections should be \textbf{kept as is}. They are used for
% automatic document processing and are highly recommended by the programme.}

% \textit{When in doubt, always consult the official template and programme guide!}

\section{Excellence \mscatag{REL-EVA-RE}}
\label{sc:excellence}

\subsection{Quality and pertinence of the project's research and innovation objectives}
\label{ssc:excellence:quality}

Proof assistants --- also called \textbf{interactive theorem provers (ITPs)} --- are software tools
used to rigorously verify formal modelling and reasoning. Contemporary systems such as \emph{Rocq},
\emph{Lean}, and
\emph{Isabelle}\footcite{the_rocq_development_team_2025_15149629,10.1007/978-3-030-79876-5_37,nipkow2002isabelle}
offer powerful frameworks for constructing formal specifications and proofs: they have been used
successfully in various applications, ranging from the verification of advanced theorems in
mathematics to the certification of complex software artifacts such as programming language
compilers, operating system kernels and cryptographic
protocols\footcite{gonthierFormalProofFour2008,leroyFormalVerificationRealistic2009,kleinSeL4FormalVerification2009,Barthe2014}.
Yet they remain notoriously difficult to learn and use, limiting broader adoption in various
settings where they could bring transformative societal impact such as \emph{mathematics education},
\emph{formal verification} and \emph{AI safety}.

The experienced researcher's (ER) work tackles the problem of usability from an original angle,
centering around \textbf{human interaction}. Building on \emph{Proof-by-Pointing} and
\emph{Proof-by-Linking}\footcite{PbP,Chaudhuri2013}, the \emph{Proof-by-Action} (PbA) paradigm
introduced in the ER's PhD thesis enables the construction of proofs by directly manipulating
logical statements in a \emph{graphical user interface} (GUI) via natural gestures like
\emph{pointing} and \emph{drag-and-drop}. This is still rooted in rigorous \textbf{proof theory} by
drawing on the recent methodology of \emph{deep inference}\footcite{deep_inference}, and complements
fully automated approaches by making \textbf{interactive refinement} of proofs more intuitive, thus
lowering the barrier to entry.

In this direction, the \textbf{scroll nets} introduced recently by the ER offer a new
\textbf{diagrammatic} framework rooted in Peirce's \emph{existential graphs} (EGs). One performs
proofs in EGs by rewriting diagrams in contexts of arbitrary depth
(Fig.~\ref{fig:modus-ponens-dynamic}), and scroll nets record these rewrites \emph{statically} as
arrows in a directed graph (Fig.~\ref{fig:modus-ponens-static}). This yields a representation of
proofs that subsumes the \emph{simply-typed $\lambda$-calculus} (STLC), the programming language underlying
modern ITPs.

Motivated by this discovery, the {\proj} project will pursue the following ambitious objective:

\setlength{\fboxsep}{5pt}
\fullwidthbox{\textit{ 
    Establish \textbf{scroll nets} as the foundation for a new generation of \textbf{ITPs}
    that support \textbf{interactive refinement} of formal specifications, proofs and programs through
    	\textbf{diagrammatic manipulations} in a GUI.
}}


Given the early stage of the theory of scroll nets and the expertise of the Theory of Computation
group at Birmingham hosting the ER, the {\proj} project will progress towards this vision by
focusing on the following Key Objectives:

\setlength{\fboxsep}{5pt}
\fullwidthbox{
    \begin{description}
        \item[\KO{1}.] extend the theory of scroll nets to account for \textbf{richer logics},
        beyond minimal implicative logic;
        \item[\KO{2}.] find natural and efficient \textbf{translations} of state-of-the-art (SOTA)
        proof systems and programming languages into scroll nets.
    \end{description}
}

\begin{figure}
  \captionsetup[subfigure]{justification=centering}
  \centering
  \begin{subfigure}[b]{0.64\textwidth}
    \begin{mathpar}
      \includepdf{2cm}{modusponens1}
      ~~~\xstep{\mathsf{Deit}}~~~
      \includepdf{2cm}{modusponens2}
      ~~~\xstep{\mathsf{Close}}~~~
      \includepdf{1.25cm}{modusponens3}
      ~~~\xstep{\mathsf{Del}}~~~
      \includepdf{0.25cm}{modusponens4}
    \end{mathpar}
    \caption{Dynamic rewriting from premiss $a \wedge (a \Rightarrow b)$ to conclusion $b$}
    \label{fig:modus-ponens-dynamic}
  \end{subfigure}
  \begin{subfigure}[b]{0.35\textwidth}
    \centering
    $\includepdf{2.25cm}{modusponens-scrollnet}$
    \caption{Static representation of rules as arrows in a bigraph}
    \label{fig:modus-ponens-static}
  \end{subfigure}
  \caption{Proof of \textit{modus ponens} $a \wedge (a \Rightarrow b) \vdash b$ in scroll nets.
  Intuitionistic implication $a \Rightarrow b$ is represented topologically by a \emph{scroll}, i.e.
  two nested ellipses that intersect exactly at one point, where the outer gray \emph{outloop}
  (resp. inner white \emph{inloop}) contains the antecedant $a$ (resp. consequent $b$).}
  \label{fig:modus-ponens}
\end{figure}

\subsubsection{State-of-the-art and contributions}

Despite their potential, the adoption of ITPs is limited by both social and foundational factors.
While community building, public communication and learning resources are crucial (as shown by the
success of Lean and its Mathlib library), many researchers point to deeper issues. A key problem is
the overwhelming \textbf{bureaucracy} of formalization: formal proofs demand a level of detail that
makes them far more time-consuming than informal proofs, with proof size around
four times bigger on average\footcite{debruijn_factor}.

The primary strategy to manage this complexity is \emph{automation}, covering both the
\emph{elaboration} of specifications from informal mathematical texts and the \emph{synthesis} of
proofs/programs satisfying these specifications. While research on both fronts progresses with the
help of SOTA machine learning techniques\footcite{blaauwbroekLearningGuidedAutomated2024},
theoretical work in \textbf{type theory} remains fundamental. As the logical backbone of modern
ITPs, type theory provides the trust and reliability that distinguishes these systems from purely
probabilistic AI.

The modern version of type theory was born in the 1970s from Martin-Löf's work, following the
\textbf{Curry-Howard correspondence} (CHC) between formal proofs (in the system of \emph{natural
deduction}) and computer programs (expressed in the STLC). Despite this remarkable connection, proof
theory and type theory have been developed mostly as separate disciplines with different conceptual
apparati, preventing cross-pollination of results. This has led the ITP community to devote little
attention to \emph{structural} proof theory, the branch of proof theory which studies combinatorial
structures for representing and manipulating proofs. The latter could however bring much
improvements to the foundations of ITPs, as illustrated by the three broad challenges that motivate
its study:
\begin{description}
    \item[\CH{1}.] Challenge 1 is the fundamental problem of \textbf{proof identity}, also
    known as Hilbert's 24\textsuperscript{th} problem\footcite{strasburger-problem-2019}. It aims to
    answer the philosophical question ``what is a proof?'', and the mathematical question ``when are
    two proofs equal?''. It is thus intimately related to \emph{homotopy type theory} which also
    investigates the structure of \emph{equality}, a known weakness of many type theories.

    \item[\CH{2}.] Challenge 2 is to find proof systems for \textbf{non-standard logics} --- such as
    \emph{modal}, \emph{intermediate}, \emph{substructural} and \emph{fixpoint} logics ---
    satisfying good enough properties as to render an algorithmic treatment of these logics
    tractable. The most important property in this respect is that of \textbf{cut elimination},
    which is essential both to reduce the complexity of \emph{proof search} (proof synthesis in ITP
    terminology), and to ensure productivity of \emph{program execution} through the CHC.
    Researchers are increasingly interested in type-theoretic formulations of non-standard logics as
    they provide expressive languages for specifying behaviors of programs that go beyond pure
    functions, including effects (modal logics), resource-sensitivity (modal and substructural
    logics) and recursion (modal and fixpoint
    logics)\footcite{tangModalEffectTypes2025,marshallLinearityUniquenessEntente2022,cloustonGuardedLambdaCalculusProgramming2017}.
    
    \item[\CH{3}.] Challenge 3 is to improve the \textbf{efficiency} of computational
    procedures on proofs, but also on programs through the CHC. A well-established principle in
    computer science and software engineering is that choosing appropriate data structures for a
    problem can lead to orders-of-magnitude improvements in efficiency. Finding the right data
    structures for such general classes of objects as proofs and programs is thus a very enticing
    goal with wide implications, including faster automation in ITPs.
\end{description}

In the past decades, two families of proof formalisms have emerged to tackle these challenges:
\begin{description}
    \item \textbf{Graphical proof systems} represent proof objects as \emph{graphs} instead of
    \emph{trees} of inference rules.
    % Many programming systems incorporate graph representations of programs in their compilation
    % pipeline in order to optimize the flow of computation, including popular machine learning
    % frameworks based on neural networks such as TensorFlow and PyTorch. Indeed, graphs appear
    % naturally when one wants to abstract from the order in which operations are sequenced, to
    % leverage \emph{parallelism} or to maximize \emph{sharing} of computations.
    \emph{Proof nets}\footcite{girard-linear-1987} are one of the first graphical proof formalisms,
    meant to reduce the bureauceacy of proofs in linear logic (\CH{2}) in order to identify their
    essence (\CH{1}). Further developments stemming from proof nets like the \emph{geometry of
    interaction} and \emph{interaction nets} have found applications in program optimization for
    hardware synthesis and parallelized
    execution\footcite{ghicaGeometrySynthesisStructured2007,mackieInteractionNetImplementation2011}
    (\CH{3}). The \emph{combinatorial proofs} of Hughes are direct decendants of proof nets mainly
    concerned with \CH{1}\footcite{Hughes_2006}, and \emph{string diagrams} from category theory have
    been applied to logic and programming languages, with well-known connections to proof
    nets\footcite{piedeleuIntroductionStringDiagrams2025}.

    \item \textbf{Deep inference} generalizes Gentzen-style proof systems by allowing inference
    rules to be applied at any depth inside of a formula, rather than being restricted to its
    top-level logical connective.
    % The terminology of ``deep inference'' was proposed by Alessio Guglielmi, who invented the
    % \emph{calculus of structures} to overcome the inability of sequent calculus to capture the
    % substructural logic $\mathsf{MV}$\footcite{Guglielmi1999ACO}.
    Calculi of structures and so-called \emph{nested sequent calculi} have been introduced to give
    proof systems enjoying cut-elimination to most substructural, modal and intermediate
    logics\footcite{kuznets_maehara-style_2019,postniece_proof_2010} (\CH{2}). Deep inference has
    also been used in the study of proof complexity, providing in some cases exponential speedup
    over sequent calculus with respect to proof size, as well as quasipolynomial-time
    cut-elimination\footcite{dasRelativeProofComplexity2015a,bruscoliQuasipolynomialNormalisationDeep2016a}
    (\CH{3}). Lastly, many deep inference formalisms enjoy CHC-style interpretations with variants
    of $\lambda$-calculus, which also improve space
    efficiency\footcite{guenot_nested_2013,gundersenAtomicLambdaCalculus2013}.
\end{description}

It is in continuation of this line of research and of his PhD
thesis\footcite{donatoDeepInferenceGraphical2024} that the ER introduced in his last preprint the
graphical framework of \textbf{scroll nets}\footcite{donatoScrollNets2025}. It is based on a long
forgotten diagrammatic proof system called \emph{existential graphs} (EGs), invented by the famous
philosopher and logician Charles Sanders Peirce at the dusk of the 19\textsuperscript{th} century
--- thus predating the very existence of proof theory and computer science. Scroll nets combine the
type-theoretic methodology of internalizing inference rules inside judgments with a graphical
structure comparable to proof nets and combinatorial proofs. This graph shares the same nodes as the
propositions involved in the proof, making scroll nets a more compact representation than these
graphical proof structures, but also surprisingly a variant of the notion of \emph{bigraph}.
Bigraphs were introduced by Milner as a foundational combinatorial structure encompassing most
models of concurrent/parallel computation, including Petri nets and his own CCS and
$\pi$-calculus\footcite{milnerBigraphicalReactiveSystems2001}. Contrary to proof nets which stem
from sequent calculus, scroll nets also enjoy a notion of \emph{detour} arising from the interplay
of introduction and elimination rules, making them closer to natural deduction. Although they were
developed completely independently, EGs and natural deduction were both born from the desire to
construct a formalism that comes as close as possible to actual reasoning, and is thus more amenable
to human manipulation.

All these discoveries hint toward the potential of scroll nets as a very expressive framework for
both proof theory and type theory, unifying features found in most of the formalisms that have
emerged from the two disciplines in a natural and efficient way. The ER's vision is to exploit the
diagrammatic nature of scroll nets to redesign both the \emph{frontend} and \emph{backend} of ITPs,
making them the \emph{interaction substrate}\footcite{mackayInteractionSubstratesCombining2025} of a
new kind of GUI in the PbA paradigm. Indeed, what crucially distinguishes scroll nets from other
graphical and deep inference proof systems is their suitability to \emph{interactive manipulation}:
every inference rule possesses a natural interpretation as a \emph{gesture} on logical statements,
either \emph{pointing} at a statement to be inserted or deleted (looping arrow in
Fig.~\ref{fig:modus-ponens-static}), or (un)copying a designated statement through
\emph{drag-and-drop} (non-looping arrow in Fig.~\ref{fig:modus-ponens-static}). This natural
semantics of gestures as a means of moving objects in space is very familiar to users of modern
GUIs, and the ER believes it could lead to a \textbf{groundbreaking ``no-code'' approach to formal
specification} that is much more intuitive than current textual interfaces.
 

% At a minimum, address the following aspects:

% \begin{itemize}
%     \item Describe the quality and pertinence of the R\&I objectives; are the
%     objectives measurable and verifiable? Are they realistically achievable?

%     \item Describe how your project goes beyond the state-of-the-art, and the
%     extent to which the proposed work is ambitious.
% \end{itemize}

\subsection{Soundness of the proposed methodology}
\label{ssc:excellence:methodology}


The research objectives of {\proj} are organised into four technical work packages, detailed below.
The core theory will be developed in \WP{1} and then extended to richer logics in \WP{2}, \WP{3} and
\WP{4}, meeting \KO{1}. \WP{3} and \WP{4} will deal with crucial features found in SOTA proving and
programming systems like recursion and dependent types, contributing to \KO{2}.


    \WP{1}.~ {\proj} will start in the restricted and well-understood setting of minimal
    implicative logic, with the aim to prove formally two key meta-theoretic properties of scroll
    nets:
    \begin{itemize}
        \item \textbf{Normalization} is a standard property of structural proof systems that asserts
        the existence (and sometimes unicity) of a \emph{normal form} associated to every proof,
        often exhibited constructively by a terminating (sometimes confluent) rewriting system. The
        ER has already sketched in his preprint such a rewriting system, called \emph{detour
        elimination}. It will need a rigorous mathematical analysis to reach its definitive form and
        obtain more insights on its computational properties. It is expected that detour elimination
        will be both terminating and confluent as well as highly parallelizable, in analogy with
        normalization procedures for other graphical formalisms like proof nets and interaction
        nets.

        \item \textbf{Sequentialization} originates from proof nets: it is a fundamental theorem establishing the
        \emph{canonicity} of proof nets as a representation of proofs in (some fragments of) linear
        logic, quotienting proofs in sequent calculus modulo rule permutations. The ER has observed a
        similar phenomenon in scroll nets, where multiple \emph{proof traces} (as in
        Fig.~\ref{fig:modus-ponens-dynamic}) which only vary in the order of inferences can build the
        same scroll net (as in Fig.~\ref{fig:modus-ponens-static}). Moreover, there is a bijective
        correspondence between the inferences in a proof trace and their static representation as edges
        in the associated scroll net, which preserves enough information to enable efficient
        reconstruction of possible traces from the latter. \WP{1} will workout the details of the
        sequentialization theorem and define formally such a reconstruction algorithm, which is not
        known to have any equivalent in the proof-theoretic landscape.
    \end{itemize}
    
    \WP{2}.~ Oostra already identified how to capture intuitionistic disjunction in EGs, by considering
    a \emph{horizontal} generalization of the scroll where one can have an arbitrary number $h$ of
    inloops attached to the same outloop\footcite{oostra_graficos_2010}. The ER has performed
    preliminary experiments on a further \emph{vertical} generalization of the scroll where inloops
    can be recursively attached to other inloops, leading to a notion of $(h,v)$-scroll with $h$
    (resp. $v$) counting the maximum number of inloops in horizontal (resp. vertical) dimension.
    Then intuitionistic disjunction and falsehood correspond to the cases where $h > 1$ and $h < 1$,
    while intuitionistic \emph{subtraction} (also called \emph{exclusion} or \emph{co-implication})
    and negation correspond to the cases where $v > 1$ and $v < 1$. This is illustrated in
    Fig.~\ref{fig:subtraction}, which shows the natural encoding of the left introduction rule for
    subtraction in sequent calculus into scroll nets. \WP{2} will prove that generalized scrolls
    lead to a sound and complete characterization of intuitionistic, bi-intuitionistic and classical
    logic by considering various restrictions on their shapes, as well as the topological properties
    of the graphs of inference rules that ensue. It will also ensure that the sequentialization and
    normalization results of \WP{1} extend well to this setting, which would constitute important
    contributions to both \CH{1} and \CH{2} by providing a non-trivial equational theory for
    classical and bi-intuitionistic proofs.

    \begin{figure}
      \captionsetup[subfigure]{justification=centering}
      \centering
      \begin{subfigure}[b]{0.30\textwidth}
        $$\scalebox{0.85}{
        $\prftree[r]{$L{-}$}{\prfsummary[$\pi$]{\Gamma, A \vdash B, \Delta}}{\Gamma, A - B \vdash \Delta}$
        }$$
        $$\includepdf{4cm}{subtraction}$$
        \caption{Encoding intuitionistic subtraction}
        \label{fig:subtraction}
      \end{subfigure}
      \begin{subfigure}[b]{0.49\textwidth}
        $$\includepdf{8cm}{factorial}$$
        \caption{Encoding natural numbers and the factorial function}
        \label{fig:factorial}
      \end{subfigure}
      \begin{subfigure}[b]{0.2\textwidth}
        $$\includepdf{4cm}{modal}$$
        \caption{Proof in predicate modal logic of $\forall x. \square (p(x) \land q(x)) \vdash (\forall x. \square p(x)) \land (\forall x. \square q(x))$}
        \label{fig:modal}
      \end{subfigure}
      \caption{Extensions of scroll nets to richer logics (\WP{2} -- \WP{4})}
      \label{fig:extensions}
    \end{figure}

    \WP{3}.~ Once disjunction is properly handled by the results of \WP{2}, adding \emph{least} and
    \emph{greatest fixpoints} of propositions is a natural step to take, that would bring the
    computational power of SOTA functional programming languages. Indeed, the combination of these
    two constructs corresponds to \emph{(co-)recursive algebraic data types} through the CHC, which
    are essential to express (co-)recursive computation on (infinite) tree-structured data.
    Fig.~\ref{fig:factorial} gives a proof-of-concept of how this could be implemented in scroll
    nets through the classic example of the factorial function. The idea is to enlarge the space of
    correct proofs by allowing \emph{self-reference} in inference rules, here represented by the
    arrow that calls the \texttt{fac} scroll into itself. \WP{3} will explore this idea further as
    well as its connections to the emerging technology of \emph{cyclic proofs}, addressing both
    \CH{2} and \KO{2}. 

    \WP{4}.~ Work packages \WP{1} to \WP{3} deal with various flavours of propositional logic,
    corresponding to Peirce's system $\mathsf{Alpha}$ of EGs. However, Peirce also explored
    extensions of EGs with new diagrammatic constructs that go beyond propositional logic: system
    $\mathsf{Beta}$ used so-called \emph{lines of identity} (LoI) to capture both quantifiers and
    equality in \emph{predicate logic}; while system $\mathsf{Gamma}$ was a more experimental
    attempt at capturing both \emph{modal} and \emph{higher-order} logics with a special kind of
    epistemic negation called ``broken cut'', represented by a dashed ellipse whose content should
    be interpreted as ``possibly not true''\footcite{maGammaGraphCalculi2018}. This is illustrated
    in Fig.~\ref{fig:modal}, where the left scroll can be read classically as $\neg \exists x.
    \diamond \neg (p(x) \land q(x))$ which is classically equivalent to the premiss $\forall x.
    \square (p(x) \land q(x))$: the LoI connecting $p$, $q$ and the outloop represents the
    universally quantified variable $x$, while the dashed inloop represents necessity. \WP{4} will
    explore the precise inference rules governing these constructs, trying to accomodate the partial
    descriptions in Peirce's writings to the modern CHC conception of {\proj}. It is expected that LoI
    can express \emph{dependent} types and \emph{identity} types, while broken cuts could capture
    either type-theoretic \emph{modalities} or type-theoretic \emph{universes} (or both), thus
    contributing greatly to \KO{1}. \WP{4} will also be essential to reach the necessary expressive
    power to simulate the \emph{calculus of inductive constructions} --- the type theory underlying
    the leading proof assistants Rocq and Lean --- and thus situate scroll nets as a realistic
    foundation for ITPs.

    % \WP{5}.~~ The Functional Machine Calculus (FMC) and Relational Machine Calculus (RMC) are two
    % foundational models of computation introduced very recently by W. Heijltjes, also co-inventor of
    % intuitionistic combinatorial
    % proofs\footcite{heijltjesFunctionalMachineCalculus2023,barrettRelationalMachineCalculus2024,heijltjes_intuitionistic_2019}.
    % They achieve respectively a unification of the two main paradigms of declarative programming,
    % \emph{functional} and \emph{logic} programming, with the more mainstream paradigm of
    % \emph{imperative} programming. An open problem is to find a suitable combination of the FMC and
    % RMC that could subsume all three paradigms. \WP{5} will explore the potential of scroll nets as
    % a solution in the well-typed fragment, by devising translations of the FMC and RMC that preserve
    % both its denotational and operational semantics. This will build on \WP{1} for the operational
    % semantics given by detour elimination; \WP{2} to account for non-determinism and failure with
    % $(n,1)$-scrolls for sum types ($n > 1$) and empty types ($n < 1$); \WP{3} to account for
    % recursion of the Kleene star operator with smallest and greatest fixpoints; and \WP{4} to
    % account for unification with LoI for dependent types.

\paragraph{Interdisciplinarity} By incorporating the CHC at the heart of its methodology, {\proj} is
by its very nature interdisciplinary, placing itself at the crossroad of \textbf{mathematical logic}
and \textbf{programming language theory}. Although not the focus of this project, the long-term
vision of exploiting scroll nets as a medium for powerful and intuitive GUIs in ITPs warrants future
intersections with the field of \textbf{human-computer interaction} (HCI). Through its dealing with
the very foundations of the notion of formal proof, {\proj} also exhibits a strong
\textbf{philosophical} flavor, which will be technically realized through its systematic
hermeneutics of the original writings of C. S. Peirce on EGs. Peirce was probably one of the last
polymathic genius in the history of western science, and his broad holistic vision of logic as the
formal investigation of the principles of scientific inquiry certainly influences the ER's own
vision, putting {\proj} in the realm of bordering areas of philosophy like \emph{epistemology},
\emph{semiotics} and \emph{metaphysics}.

\paragraph{Gender}
Not applicable, because the research of {\proj} is of an abstract and theoretical nature.

\paragraph{Open science and research data management}
All the results of {\proj} will be made available as arXiv pre-prints, with the aim of encouraging
informal evaluation from the scientific community. Moreover, the ER will (co-)organise a workshop to
communicate results and encourage feedback from other researchers (Sec. 2.2). The project
deliverables will be published in the proceedings of high-rated and peer-reviewed conferences and in
scientific journals (Sec.~\ref{ssc:impact:outcomes}). Proceeding publications will be made
available on arXiv, and journal contributions will be published in Open Access format, for which
funding is foreseen. {\proj} does not envisage the collection of any kind of data, and eventual
software prototypes experimenting with an implementation of scroll nets will all be open-sourced.

% At a minimum, address the following aspects:

% \begin{itemize}
%     \item \emph{Overall methodology}: Describe and explain the overall methodology,
%     including the concepts, models and assumptions that underpin your work.
%     Explain how this will enable you to deliver your project’s objectives. Refer
%     to any important challenges you may have identified in the chosen methodology
%     and how you intend to overcome them.

%     \item \emph{Integration of methods and disciplines to pursue the objectives}:
%     Explain how expertise and methods from different disciplines will be brought
%     together and integrated in pursuit of your objectives. If you consider that
%     an inter-disciplinary\footnotemark{} approach is unnecessary in the
%     context of the proposed work, please provide a justification.

%     \footnotetext{\emph{Interdisciplinarity} means the integration of information,
%     data, techniques, tools, perspectives, concepts or theories from two or
%     more scientific disciplines.}

%     \item \emph{Gender dimension and other diversity aspects}: Describe how the
%     gender dimension and other diversity aspects are taken into account in the
%     project's research and innovation content. If you do not consider such a
%     gender dimension to be relevant in your project, please provide a
%     justification.
%     \begin{itemize}
%         \item Remember that that this question relates to the \emph{content} of the
%         planned research and innovation activities, and not to gender balance
%         in the teams in charge of carrying out the project.

%         \item  Sex, gender and diversity analysis refers to biological
%         characteristics and social/cultural factors respectively. For guidance on
%         methods of sex / gender analysis and the issues to be taken into account,
%         please refer to this
%         \href{https://op.europa.eu/en/publication-detail/-/publication/33b4c99f-2e66-11eb-b27b-01aa75ed71a1/language-en}{link}.
%     \end{itemize}

%     \item \emph{Open science practices}: Describe how appropriate open science
%     practices are implemented as an integral part of the proposed methodology.
%     Show how the choice of practices and their implementation is adapted to the
%     nature of your work in a way that will increase the chances of the project
%     delivering on its objectives \emph{[1/2 page]}. If you believe that none of
%     these practices are appropriate for your project, please provide a justification
%     here.

%     \emph{Open science is an approach based on open cooperative work and systematic
%     sharing of knowledge and tools as early and widely as possible in the process.
%     Open science practices include early and open sharing of research (for example
%     through pre-registration, registered reports, pre-prints, or crowd-sourcing);
%     research output management; measures to ensure reproducibility of research
%     outputs; providing open access to research outputs (such as publications,
%     data, software, models, algorithms, and workflows); participation in open
%     peer-review; and involving all relevant knowledge actors including citizens,
%     civil society and end users in the co-creation of R\&I agendas and contents
%     (such as citizen science).}

%     \emph{Please note that this does not refer to outreach actions that may be
%     planned as part of the communication, dissemination and exploitation activities.
%     These aspects should instead be described below under ``Impact''.}

%     \item Proposals selected for funding under Horizon Europe will need to develop
%     a detailed data management plan (DMP) for making their data/research outputs
%     findable, accessible, interoperable and reusable (FAIR) as a deliverable by
%     month 6 and revised towards the end of a project's lifetime. The DMP should
%     describe how research outputs (especially research data) generated and/or
%     collected during the project will be managed so as to ensure that they are
%     findable, accessible, interoperable and reusable.

%     \emph{For guidance on open science practices and research data management,
%     please refer to the relevant section of the
%     \href{https://ec.europa.eu/info/funding-tenders/opportunities/docs/2021-2027/horizon/temp-form/af/af_he-msca-pf_en.pdf}{HE Programme Guide}
%     on the Funding \& Tenders Portal.}
% \end{itemize}

\subsection{Quality of the supervision, training and of the two-way transfer of
    knowledge between the researcher and the host}
\label{ssc:excellence:supervision}

\paragraph{Quality of the supervision}
The project will be supervised by Anupam Das, associate professor at the School of Computer Science
of the University of Birmingham (UoB) in the Theory of Computation (ToC) group. Prof. Das is a leading
expert in proof theory, who is nowadays particularly active in the area of cyclic proofs for
fixpoint logics (\WP{3}). He has a strong background in deep inference, linear logic and their
applications to complexity theory, and has also made multiple contributions to the study of
intuitionistic modal logics (\WP{4}). His supervision will thus be essential to the success of
{\proj}.

Prof. Das has an excellent record of almost 60 publications in top-tier journals (LMCS, JAR) and
peer-reviewed conference proceedings (LICS, CSL, FSCD, IJCAR, TABLEAUX) in logic, theoretical
computer science and automated theorem proving. He also has a good track record in mentoring young
researchers: he co-supervised 2 Ph.D. students with Prof. Escardo and Prof. Paul Levy --- with whom
the ER expects collaborations on \WP{4} and \WP{5} --- and supervised 5 postdocs thanks to the
funding that he obtained as a recipient of the UKRI Future Leaders Fellowship \textit{Structure vs.
Invariants in Proofs} (2020--2024, renewed for 2024--2027). He was himself an MSCA postdoctoral
fellow from 2017 to 2019.

\paragraph{Training}
The ER's overarching career goal is to secure a computer science professorship in Europe. In the
medium term, he aspires to obtain an unsupervised temporary position and funding by an ERC starting
grant. The acquisition of a prestigious MSCA grant would significantly enhance his prospects. The
proposed training program is structured into three Training Objectives:
\begin{description}
    \item[\TO{1}.] \textbf{Gaining scientific knowledge:}
    during his tenure at the host institution, the ER will expand his scientific horizon by
    immersing himself in the English school of proof theory, the historical centre for the
    development of the deep inference methodology. Here, he will gain further knowledge of advanced
    SOTA topics such as cyclic proofs, intuitionistic modal logics and homotopy type
    theory. At least weekly meetings with his supervisor will serve as forums to assess project
    progress, explore scientific inquiries, and exchange innovative ideas. The ER will also actively
    participate in the ToC's weekly seminar series, featuring speakers from renowned global research
    institutions.

    \item[\TO{2}.] \textbf{Enhance management, leadership and grant writing skills:}
    the ER will participate in several management and leadership skills courses offered by UoB's
    People and Organization Development (POD) division, including Research Team Leader, Effective
    Financial Management, and other programs on Health \& Safety in the Workplace, Ethics \&
    Governance in Research, Scientific Communication, Public Engagement \& Dissemination, and
    Personal Effectiveness. The ER will also attend the POD Grant Writing workshop. These skills
    will enrich the ER's ability to adapt to changes and undertake a variety of activities to
    support his high-level interdisciplinary research capability.

    \item[\TO{3}.] \textbf{Enhance supervision and teaching skills:}
    the ER will actively engage in mentoring the supervisor's students, providing individual
    guidance on their research projects, and offering courses in specialised subjects such as
    structural proof theory and interactive theorem proving. He will also take on responsibilities
    such as crafting research internship proposals and overseeing interns during 2-3 month periods
    of study.

\end{description}

\textbf{Applicant-to-host transfer of knowledge}~ The project will greatly benefit from the combined
expertise of several leading researchers at ToC. For \WP{2}, the ER will collaborate with Dr.
Shillito, a postdoctoral fellow with internationally recognized expertise in the syntax and
semantics of bi-intuitionistic logic. For \WP{4}, the ER will also collaborate with Dr. Marin and
Prof. Escardo, eminent specialists in modal logics and dependent types/homotopy type theory/ITPs,
respectively. For the foundational work in \WP{1}, the ER may collaborate with Prof. Ghica, who has
been applying graphical formalisms to model computations for several decades in both academia and
industry.
% For \WP{5}, the ER will additionally collaborate with Prof. Paul Levy, a renowned expert in
% effectful programming languages. His groundbreaking work in this field was recognized with the 2025
% Alonzo Church Award for Outstanding Contributions to Logic and Computation.

\paragraph{Applicant-to-host transfer of knowledge}
The ER will bring to ToC his worldwide unique expertise on EGs and their proof theory, as well as
the application of deep inference and graphical proof theory to interactive theorem proving. He will
share his knowledge through one-to-one interactions with ToC members and students, by speaking at
seminars, by supervising Master projects and by (co-)organising a workshop on {\proj} topics
(Sec.~\ref{ssc:impact:outcomes}).

\subsection{Quality and appropriateness of the researcher's professional
    experience, competences and skills}
\label{ssc:excellence:researcher}

The ER is an internationally recognised expert in proof theory and interactive theorem proving. His
PhD thesis, defended in May 2024, revolves around the design and implementation of novel proof
systems that enable the construction of formal proofs through direct manipulation in a GUI. In
particular, Chapters 9 and 10 provide a deep 100 pages analysis of the proof theory of EGs both in
propositional/first-order and intuitionistic/classical logic, and describe a prototype of GUI for
ITPs based on this theory called Flower Prover, which is publicly accessible and usable
online\footcite{flower-prover}. Thus the {\proj} project constitutes a direct continuation and
substantial extension of the ER's doctoral research and benefits from his personal interdisciplinary
background in computer science, mathematics and philosophy. Indeed, although the ER has been mainly
trained in computer science during his studies, he also attended the 1\textsuperscript{st} year of
the Université Paris 1's master in Logic and Philosophy of Science (LOPHISC). After finishing his
PhD, the ER has secured a 1-year postdoctoral position at the Grothendieck Institute, working with
Olivia Caramello and Fields medalist Laurent Lafforgue on the formalization of topos theory in the
Lean proof assistant, thus broadening significantly his mathematical knowledge. Topos theory also
has well-known applications to the semantics of higher-order and dependent type theories which have
been investigated by researchers of ToC like Steve Vickers, making this acquired knowledge relevant
to \WP{4} of {\proj}.

The ER has proven his potential to carry out interesting and impactful research witnessed by two
publications at top-tier conferences (CPP, FSCD), as well as his strong independence by being the
sole author of his second article. He is an experienced speaker, having delivered ? conference
talks, ? workshop talks, and ? invited seminar talks to date. Moreover, he has been invited by
Ahti-Veikko Pietarinen --- one of the most prolific Peirce scholars in the world with an h-index of
35 --- to contribute a chapter to the forthcoming book \textit{Peirce’s Philosophy of Notation} in
the Peirceana
series\footnote{\url{https://www.amazon.fr/Peirces-Philosophy-of-Notation/dp/3110649500}},
demonstrating recognition of his interdisciplinary and international expertise on EGs. The ER has
been involved in teaching during every year of his PhD: he taught various courses on programming
languages at undergraduate and graduate levels, both at Université Paris 7 and École Polytechnique.
This involved teaching classes to groups of 50 students and performing duties such as exam grading,
project review and practical sessions mentoring.

% The ER has demonstrated his ability to secure funding by obtaining his PhD scholarship from École
% Polytechnique. Drawing from the extensive experience of the supervisor and fellow members of the ToC
% group in securing grants, the ER will be well-equipped for future endeavours, such as pursuing an
% ERC starting grant.

\section{Impact \mscatag{IMP-ACT-IA}}
\label{sc:impact}

\subsection{Credibility of the measures to enhance the career perspectives
    and employability of the researcher and contribution to his/her skills
    development}
\label{ssc:impact:career}

The MSCA fellowship is a crucial step towards the ER's long-term goal of securing a professorship in
Europe. It will directly support his medium-term ambition of obtaining an ERC starting grant by
providing the ideal environment to mature his research programme on scroll nets from a promising
initial discovery into a cornerstone of next-generation ITPs. The project will significantly enhance
his academic track record through high-impact publications and presentations at top-tier venues,
demonstrating his capacity for independent and groundbreaking research.

% This fellowship will serve as a catalyst for the ER's scientific growth. By joining the ToC group at
% UoB, a world-leading centre for proof theory, he will deepen his expertise in areas directly
% relevant to {\proj}, such as cyclic proofs, modal logics, and homotopy type theory (\TO{1}). This
% new knowledge will complement his unique background in EGs and interactive theorem proving, creating
% a powerful synthesis. The training in management, leadership, and grant writing (\TO{2}), combined
% with the experience of managing the project's work packages and budget, will equip him with the
% necessary skills for leading a research group. Furthermore, his teaching and supervision activities
% (\TO{3}) will prepare him for future mentorship roles as a professor.

The fellowship at UoB offers unparalleled networking opportunities. The ER will be integrated into
the vibrant ToC group, fostering daily interactions and collaborations with his supervisor and its
members, whose expertise is vital for the success of all five work packages (see
Sec.~\ref{ssc:excellence:supervision}). More generally, the UK is known to be one of the epicenters
for research in proof theory and programming language theory, with leading researchers in these two
fields at top universities; this includes the \emph{Mathematical foundations of computation} group
at University of Bath, the \emph{Programming Principles, Logic, and Verification} group at
University College London, and the \emph{Programming, Logic, and Semantics} group at University of
Cambridge. This strategic positioning will expand the ER collaborative network, solidifying his
standing in the international community and paving the way for future joint projects. The outcomes
of {\proj}, disseminated through open-access publications, pre-prints, and a dedicated workshop
(Sec.~\ref{ssc:impact:outcomes}), will cement the ER's reputation as a leader in the field, boosting
his employability and career prospects.

\subsection{Suitability and quality of the measures to maximise expected
    outcomes and impacts, as set out in the dissemination and exploitation plan,
    including communication activities \mscatag{COM-DIS-VIS-CDV}}
\label{ssc:impact:outcomes}

\paragraph{Scientific events}
The results of the {\proj} project are planned to be published in high-impact journals such as LMCS
and TCS, and in the proceedings of conferences like LICS, CSL, FSCD, FroCoS, TABLEAUX and DIAGRAMS.
All of these are international top-tier venues focusing on logical topics in computer science, to
the exception of DIAGRAMS which is interdisciplinary and open to all fields that study diagrammatic
notations, including mathematics, cognitive science, psychology, philosophy and linguistics.
DIAGRAMS 2026 will be hosted by Pawel Sobocinski's group in Tallinn, internationally renowned for
their expertise in string diagrams, with Francesco Bellucci --- co-editor with Pietarinen of the
Peirceana book series (see Sec.~\ref{ssc:excellence:researcher}) --- as program chair. FroCoS is
also specifically interested in frameworks that can combine multiple logics and thus particularly
relevant to {\proj}. Additionally, the ER will report about the status of the {\proj} project
frequently at international workshops like LFMTP or TYPES. He will propose to co-organize with
colleagues Dr. Haydon (expert in EGs) and Dr. Di Giorgio (expert in string diagrams) a new
``Graphical Logic Workshop'' (GLW) affiliated with a major conference, filling the absence of a
federated community of researchers on {\proj} topics. All these measures will enable a wide range of
possible future research to be conducted by contributors, external collaborators, and independent
research groups.

\paragraph{Public communication}
As the ultimate goal of {\proj} is to propose a new notation for logic and programming that is more
accessible to non-experts, the ER will undertake multiple initiatives to engage audiences outside of
academia. A personal blog dedicated to scroll nets will be created to explain the basic principles
of the formalism, assuming little to no background in logic and programming. This will serve as a
direct test for the accessibility and intuitiveness of the notation, and may include interactive
animations in the web browser inspired by the popular online learning platform Brilliant.org, whose
slogan ``Learn by doing'' fits nicely with the ER's Proof-by-Action paradigm. To reach a more
established audience, the ER will also contribute to Prof. Anupam Das's existing proof theory
blog\footnote{\url{https://prooftheory.blog/about-the-proof-theory-blog/}}. Furthermore, a YouTube
channel may be launched to explore the concepts of visual logic and programming, leveraging the
video format to provide dynamic and engaging explanations. Finally, the ER will actively participate
in science popularization events such as \emph{The Big Bang
Fair}\footnote{\url{https://www.thebigbang.org.uk/}} in Birmingham and \emph{Computer Science in
Action} in
London\footnote{\url{https://educationinaction.org.uk/study-day/computer-science-in-action-27-11-2024/}},
presenting the project's ideas to students, teachers, and the general public to inspire the next
generation of computer scientists.



% At a minimum, address the following aspects:

% \begin{itemize}
%     \item \emph{Plan for the dissemination and exploitation activities, including
%     communication activities}\footnotemark{}: Describe the planned measures to
%     maximize the impact of your project by providing a first version of your ‘plan
%     for the dissemination and exploitation including communication activities’.
%     Describe the dissemination, exploitation measures that are planned, and the target
%     group(s) addressed (e.g. scientific community, end users, financial actors,
%     public at large). Regarding communication measures and public engagement
%     strategy, the aim is to inform and reach out to society and show the activities
%     performed, and the use and the benefits the project will have for citizens.
%     Activities must be strategically planned, with clear objectives, start at the
%     outset and continue through the lifetime of the project. The description of
%     the communication activities needs to state the main messages as well as the
%     tools and channels that will be used to reach out to each of the chosen target
%     groups.

%     \footnotetext{In case your proposal is selected for funding, a more detailed
%     Dissemination and Exploitation plan will need to be provided as a mandatory
%     project deliverable during project implementation.}

%     \item \emph{Strategy for the management of intellectual property, foreseen
%     protection measures}: if relevant, discuss the strategy for the management
%     of intellectual property, foreseen protection measures, such as patents,
%     design rights, copyright, trade secrets, etc., and how these would be used
%     to support exploitation.

%     \item All measures should be proportionate to the scale of the project, and
%     should contain concrete actions to be implemented both during and after the
%     end of the project.
% \end{itemize}

\subsection{The magnitude and importance of the project’s contribution to the
    expected scientific, societal and economic impacts}
\label{ssc:impact:future}

The core vision of {\proj} is to make formal reasoning more accessible. If successful, its impact
will extend far beyond the project's immediate scope, transforming domains where rigorous
verification is critical but currently hindered by the steep learning curve of ITPs.

\begin{description}
    \item \textbf{Scientific/Societal Impact in Mathematics Education:}~ {\proj} will provide a ``no-code''
    platform for students to interactively explore mathematical concepts. The diagrammatic nature of
    scroll nets offers a more intuitive way to grasp abstract proofs than traditional symbolic
    logic, fostering deeper understanding and engagement\footcite{minhResearchReportProof2024}. This
    could democratize access to advanced mathematics and computer science, benefiting students and
    educators from secondary to higher education by providing a collaborative and visual learning
    medium.

    \item \textbf{Economic/Technological Impact in Formal Verification:}~ By lowering the barrier to entry,
    {\proj} will enable wider adoption of formal methods in industry. The PbA paradigm will allow
    domain experts (e.g., engineers, systems architects) to directly participate in the verification
    process without extensive training in ITPs. This will reduce costs and development time while
    increasing the reliability of safety-critical systems in sectors like aerospace, automotive, and
    healthcare, leading to significant economic and safety benefits.

    \item \textbf{Scientific/Technological Impact in AI Safety:}~ {\proj} will contribute to the development
    of more reliable neurosymbolic AI systems. By providing a transparent and verifiable logical
    reasoning component, scroll nets can complement the probabilistic nature of LLMs, mitigating the
    risk of ``hallucinations'' and ensuring that AI-driven scientific discovery is grounded in
    rigorous, auditable proof. This enhances the trustworthiness of AI systems in high-stakes
    research and decision-making.
\end{description}

\mscatagend{COM-DIS-VIS-CDV} \mscatagend{IMP-ACT-IA}

\section{Quality and Efficiency of the Implementation
         \mscatag{QUA-LIT-QL} \mscatag{WRK-PLA-WP} \mscatag{CON-SOR-CS} \mscatag{PRJ-MGT-PM}}
\label{sc:implementation}

\subsection{Quality and effectiveness of the work plan, assessment of risks and
    appropriateness of the effort assigned to work packages}
\label{ssc:implementation:workplan}

The project is structured into four technical \textbf{work packages} (\WP{1}-\WP{4}) to be completed
over the 24-month fellowship. The work plan detailed in Fig.~\ref{fig:gantt} is designed to
front-load the foundational theoretical work in the first year, allowing the second year to focus on
extensions, applications, and dissemination. Each technical work package has been divided into
several \textbf{deliverables}, corresponding to parts of the development of {\proj} that have
potential for publication as individual conference articles. An additional deliverable (\DE{11})
will be the creation of the project website/blog for public communication. \textbf{Key milestones}
with regard to applications to ITPs are the completion of the foundational theory (\MI{1}), the
successful extension to fixpoint logics (\MI{2}), and the development of the theory for dependent
types (\MI{3}), corresponding respectively to \WP{1}, \WP{3}.a and \WP{4}.a.

\paragraph{Risks and Mitigation}
Potential risks to the project have been identified and mitigation strategies are in place to ensure
its successful completion (Table~\ref{tbl:risks}). Notably, the ER does not expect to realistically
achieve the writing of 10 papers (deliverables \DE{1} to \DE{10}) in 2 years. Rather, this should be
considered as a strategy to secure multiple independent avenues for research, where focus on
deliverables marked as \emph{optional} can be released in case of roadblocks or failures. Several
deliverables in the same work package might also be combined into a single long journal paper.
Regular meetings with the supervisor will allow for continuous risk assessment.

\begin{figure}
    \centering

    \begin{ganttchart}[
        hgrid,vgrid,
        expand chart=\textwidth,
        x unit=1cm,
        y unit title=0.8cm,
        y unit chart=0.35cm,
        % milestone/.append style={shape=star, star point ratio=2.5, fill=yellow},
        % milestone height=1.5,
        milestone inline label node/.append style={left=5mm},
    ]{1}{24}
        % \gantttitle{Year 1}{12}
        % \gantttitle{Year 2}{12} \\
        \gantttitlelist{1,...,24}{1} \\

        \ganttbar[name=wp1-a, bar/.append style={fill={rgb,255:red,255; green,170; blue,0}}]{\WP{1}.a}{1}{6} \\
        \ganttbar[name=wp1-b, bar/.append style={fill={rgb,255:red,255; green,170; blue,0}}]{\WP{1}.b}{1}{6} \\

        \ganttbar[name=wp2-a, bar/.append style={fill={rgb,255:red,226; green,120; blue,255}}]{\WP{2}.a}{7}{9} \\
        \ganttbar[name=wp2-b, bar/.append style={fill={rgb,255:red,226; green,120; blue,255}, dashed}]{\WP{2}.b}{9}{12} \\
        \ganttbar[name=wp2-c, bar/.append style={fill={rgb,255:red,226; green,120; blue,255}, dashed}]{\WP{2}.c}{10}{12} \\

        \ganttbar[name=wp3-a, bar/.append style={fill={rgb,255:red,98; green,173; blue,255}}]{\WP{3}.a}{13}{16} \\
        \ganttbar[name=wp3-b, bar/.append style={fill={rgb,255:red,98; green,173; blue,255}, dashed}]{\WP{3}.b}{15}{18} \\

        \ganttbar[name=wp4-a, bar/.append style={fill={rgb,255:red,100; green,255; blue,100}}]{\WP{4}.a}{17}{20} \\
        \ganttbar[name=wp4-b, bar/.append style={fill={rgb,255:red,100; green,255; blue,100}, dashed}]{\WP{4}.b}{19}{24} \\
        \ganttbar[name=wp4-c, bar/.append style={fill={rgb,255:red,100; green,255; blue,100}, dashed}]{\WP{4}.c}{21}{24} \\

        \ganttbar[name=website, bar/.append style={fill={rgb,255:red,200; green,200; blue,200}}]{Website}{1}{24} \\

        \ganttnewline
        \ganttmilestone[inline]{\MI{1}}{6}
        \ganttmilestone[inline]{\MI{2}}{16}
        \ganttmilestone[inline]{\MI{3}}{20}

        \ganttlink{wp1-a}{wp2-a}
        \ganttlink{wp1-a}{wp2-b}
        \ganttlink{wp1-a}{wp2-c}
        \ganttlink{wp1-a}{wp3-a}
        \ganttlink{wp1-a}{wp3-b}
        \ganttlink{wp1-a}{wp4-a}
        \ganttlink{wp1-a}{wp4-b}
        \ganttlink{wp1-a}{wp4-c}
        \ganttlink{wp2-a}{wp2-b}
        \ganttlink{wp2-a}{wp2-c}
        \ganttlink{wp3-a}{wp3-b}
        \ganttlink{wp4-a}{wp4-c}
    \end{ganttchart}

    \vspace{1em}
        \renewcommand{\arraystretch}{0.8} % reduce the vertical space between rows in tables
    \begin{tabular}{c p{0.65\textwidth} p{0.05\textwidth} c}
        \fontsize{11pt}{10pt}\selectfont
        \textit{Tasks} & \textit{Description} & \textit{Deliv.} & \textit{Month} \\
        \midrule
        \TaskCircle{255}{170}{0} \WP{1}.a & Sequentialization for minimal implicative logic & \DE{1} & 6 \\
        \TaskCircle{255}{170}{0} \WP{1}.b & Normalization for minimal implicative logic & \DE{2} & 6 \\
        \TaskCircle{226}{120}{255} \WP{2}.a & Extension to intuitionistic propositional logic with $(n,1)$-scrolls & \DE{3} & 9 \\
        \TaskCircle{226}{120}{255} \WP{2}.b & Extension to bi-intuitionistic propositional logic with $(n,m)$-scrolls & \DE{4}$\mathsf{o}$ & 12 \\
        \TaskCircle{226}{120}{255} \WP{2}.c & Equational theory of classical proofs through topology of rules & \DE{5}$\mathsf{o}$ & 15 \\
        \TaskCircle{98}{173}{255} \WP{3}.a & Extension to least fixpoints of propositions & \DE{6} & 16 \\
        \TaskCircle{98}{173}{255} \WP{3}.b & Extension to least+greatest fixpoints of propositions & \DE{7}$\mathsf{o}$ & 18 \\
        \TaskCircle{100}{255}{100} \WP{4}.a & Extension to first-order predicate logic ($\mathsf{Beta}$) & \DE{8} & 20 \\
        \TaskCircle{100}{255}{100} \WP{4}.b & Extension to the intuitionistic modal cube & \DE{9}$\mathsf{o}$ & 22 \\
        \TaskCircle{100}{255}{100} \WP{4}.c & Extension to higher-order logic and dependent types ($\mathsf{Gamma}$) & \DE{10}$\mathsf{o}$ & 24 \\
    \end{tabular}

    \caption{\textbf{Above:} Gantt chart summarising the work plan of {\proj}. \textbf{Below:} tasks
    and associated deliverables. Deliverables marked with the letter $\mathsf{o}$ (drawn with a
    dashed contour) are considered optional.}
    \label{fig:gantt}
\end{figure}

\begin{table}[h!]
\centering
\fontsize{11pt}{10pt}\selectfont
\caption{Risk Assessment and Mitigation Plan}
\begin{tabular}{p{0.4\textwidth} p{0.4\textwidth} c}
\toprule
\textbf{Risk} & \textbf{Mitigation Strategy} & \textbf{Severity} \\
\midrule
\textbf{R1:} Foundational theory in \WP{1} (normalization/sequentialization) proves more complex than anticipated, delaying subsequent $\mathsf{WP}$s. & Allocate more time for \WP{1}. If major issues arise, consult with external experts in graphical proof theory, such as Dr. Heijltjes. & Medium \\
\addlinespace
\textbf{R2:} The proposed extensions in \WP{2}-\WP{3} do not yield sound and complete systems or fail to extend the core meta-theorems. & The expertise of the supervisor (cyclic proofs) and collaborators (modal/bi-intuitionistic logic) will be leveraged. If a specific extension fails, focus will shift to the more promising ones. & Medium \\
\addlinespace
\textbf{R3:} Failure to extend the theory to dependent types in \WP{4}. & Intensify collaboration with Prof. Escardo, an expert in dependent types. If significant roadblocks appear, focus will shift to the less speculative modal logic aspects of \WP{4}. & Medium \\
\bottomrule
\end{tabular}
\label{tbl:risks}
\end{table}

\subsection{Quality and capacity of the host institutions and participating
organisations, including hosting arrangements}
\label{ssc:implementation:host}

\subsubsection{Quality and capacity of the host institution}

The University of Birmingham (UoB) provides an ideal environment for {\proj}, being a leading
research institution ranked 76th globally in the 2025 QS World University Rankings. Its Theory of
Computation (ToC) group is a world-renowned centre for theoretical computer science, featuring
prominent researchers like Paul Blain Levy (2025 Alonzo Church awardee), Martin Escardo, and the
supervisor, Anupam Das. The university's robust infrastructure and demonstrated success in hosting
MSCA Fellows (see Part B-2, Section 5) ensure comprehensive support for the project. UoB's
administrative services, including a dedicated Horizon Europe support team, will manage all
financial and contractual aspects, from the Grant Agreement to final reporting, allowing the ER to
focus on research. The project will adhere to the European Code of Conduct for Research Integrity,
with clear protocols for scientific conduct and ethics overseen by the supervisor.

\subsubsection{Hosting arrangements}

The ER will be fully integrated as an employee at UoB, benefiting from its established support
system for MSCA Fellows. This includes access to university services, health and safety resources,
and professional development opportunities through the People and Organisational Development (POD)
and the Centre for Learning and Academic Development (CLAD). The ER's professional growth will be
supported by Personal Development Reviews (PDRs), ensuring alignment with career objectives. UoB is
committed to upholding the principles of the European Charter for Researchers and the HR Excellence
in Research Award.

Hosted within the ToC group, the ER will join a dynamic community of over 40 researchers, including
faculty, postdoctoral fellows, and PhD students. This environment will foster intellectual exchange
and collaboration. The group's weekly seminars and working groups will provide a platform for
presenting progress on {\proj}, refining milestones, and preparing for dissemination activities.
Regular meetings with the supervisor will ensure close guidance and mentorship. The School of
Computer Science will provide all necessary resources, including dedicated office space,
high-performance computing, and full access to digital libraries and scientific publications from
all major publishers.

% At a minimum, address the following aspects:

% \begin{itemize}
%     \item Hosting arrangements, including integration in the team/institution(s)
%     and support services available to the researcher.

%     \item Quality and capacity of the participating organisations, including
%     infrastructure, logistics and facilities. Additional information should be
%     outlined in Part B-2 Section 5 (``Capacity of the Participating Organisations'').
% \end{itemize}

% Note that for GF, both the quality and capacity of the outgoing Third Country
% host and the return host should be outlined.

% \subsection*{Associated partners linked to a beneficiary {\protect\footnotemark{}}}
% \label{ssc:implementation:partners}

% If applicable, outline here the involvement of any 'associated partners linked
% to a beneficiary' (in particular, the name of the entity, the type of link with
% the beneficiary and the tasks to be carried out).

% \footnotetext{See the definitions section of the MSCA Work Programme for
% further information.}

\mscatagend{CON-SOR-CS} \mscatagend{PRJ-MGT-PM} \mscatagend{QUA-LIT-QL} \mscatagend{WRK-PLA-WP}
\end{document}

% kate: default-dictionary en_GB;
