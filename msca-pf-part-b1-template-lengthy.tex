\documentclass[12pt,draftproposal]{msca-pf}
% {{{ packages

\usepackage[left=15mm,right=15mm]{geometry}

% better environments
\usepackage[nameinlink, noabbrev]{cleveref}
\usepackage[shortlabels]{enumitem}
\usepackage{booktabs}
\usepackage{caption}
\usepackage{subcaption}

% better typography
\usepackage[activate={true,nocompatibility}, % activate protrusion and font expansion
            final,              % enable microtype, use draft to disable
            tracking=true,
            factor=1100,        % more protrusion
            stretch=10,         % smaller values (default 20, 20) to avoid blurring
            shrink=10]{microtype}
\SetTracking{encoding={*}, shape=sc}{40}

% bibliography
\usepackage[style=mla,backend=biber]{biblatex}

% }}}

% {{{ formatting

% NOTE: this needs to be kept so that the sections match the template!
\setcounter{section}{0}

% }}}

% {{{ commands

% add your own fancy commands
% \NewDocumentCommand \mycmd { m } {\textbf{#1}}

\newcommand{\fullwidthbox}[1]{%
    \noindent\makebox[\textwidth]{%
        \fbox{%
            \begin{minipage}{\dimexpr\textwidth-2\fboxsep-2\fboxrule\relax}
                \centering
                #1
            \end{minipage}%
        }%
    }%
}

% }}}

% {{{ title

\title{Part B-1}
\author{Pablo Donato}
\date{\today}

% NOTE: update these as necessary
\mscaidentifier{HORIZON-MSCA-2025-PF-01}
\newcommand{\proj}{\textsc{ScrollNets}}
\mscaproject{\proj}

% }}}

% {{{ bibliography

% \usepackage{filecontents}% to embed the file `myreferences.bib` in your `.tex` file

% \begin{filecontents}{myreferences.bib}
% @online{foo12,
%   year = {2012},
%   title = {footnote-reference-using-european-system},
%   url = {http://tex.stackexchange.com/questions/69716/footnote-reference-using-european-system},
% }
% \end{filecontents}

% \addbibresource{MSCA.bib}
\bibliography{MSCA.bib}

% }}}

\begin{document}

\maketitle

% \textit{The text in each section is provided only as guidance and should be deleted
% in the final application (including this text). The tags in the section titles
% and at the end of certain sections should be \textbf{kept as is}. They are used for
% automatic document processing and are highly recommended by the programme.}

% \textit{When in doubt, always consult the official template and programme guide!}

\section{Excellence \mscatag{REL-EVA-RE}}
\label{sc:excellence}

\subsection{Quality and pertinence of the project's research and innovation objectives
    (and the extent to which they are ambitious, and go beyond the state of the art)}
\label{ssc:excellence:quality}

\subsubsection{Introduction and state of the art}

\textbf{Proof assistants} --- also called \emph{interactive theorem provers} (ITPs) --- are software
tools used to rigorously verify formal modelling and reasoning. Contemporary systems such as
\emph{Rocq}~\footcite{the_rocq_development_team_2025_15149629},
\emph{Lean}~\footcite{10.1007/978-3-030-79876-5_37}, and
\emph{Isabelle}~\footcite{nipkow2002isabelle} offer powerful frameworks for constructing formal
specifications and proofs: they have been used successfully in various applications, ranging from
the verification of advanced theorems in mathematics
\footcite{gonthierFormalProofFour2008}
to the certification of complex software artifacts such as programming language compilers
\footcite{leroyFormalVerificationRealistic2009}, operating system kernels
\footcite{kleinSeL4FormalVerification2009} and cryptographic protocols \footcite{Barthe2014}. Yet
they remain notoriously difficult to learn and use, limiting their broader adoption in various
settings where they could bring transformative societal impact; to name a few:
\begin{itemize}
    \item \textbf{mathematics education}, where they could act as a unifying medium for interactive
    exploration and understanding of mathematical concepts, fostering closer collaboration amongst
    students and offloading some teaching burden (e.g. grading) through automation;
    \item \textbf{program verification}, bringing higher standards of quality assurance (QA) to software-producing businesses, especially in safety-critical industries such as healthcare, transportation and energy;
    \item \textbf{artificial intelligence} (AI), where the uncertainty inherent to current technologies
    based on probabilistic techniques such as large language models (LLMs) could be mitigated by the
    exact logical reasoning capabilities of ITPs, an approach sometimes termed \emph{neurosymbolic
    AI}.
\end{itemize}
In view of this large potential for applications, it is natural to ask what exactly limits adoption
of the current generation of ITPs. The recent surge of interest arising both in academia and
industry --- in great part due to the popularity of the Lean language and its Mathlib library ---
suggests that social factors such as public communication, community building, and vast amounts of
expository content and learning resources all play an important role in the widespread appropriation
of this complex technology. Even more recently, the promise of a new kind of generative AI free from
so-called ``hallucinations'' that could aid in accelerating scientific discoveries has been a
powerful narrative attracting much attention and funding\footcite{TODO: reference to some media
outlet announcing some startup funding like Harmonic, or maybe to the Lean FRO roadmap}.

However, many researchers in the field agree that current ITPs suffer from more \emph{foundational}
issues that affect directly their accessibility and ease of use, as well as their ability to scale
to larger developments. While these issues are quite diverse in nature, a recurring theme since the
birth of the technology in the 60s is the overwhelming \textbf{bureaucracy} involved in
formalization efforts: formal proofs require a level of care and detail that is often far superior
and more time consuming than what is expected from informal paper proofs.
The main approach to tame this complexity has been to try and \emph{automate} the various processes
involved in formalization, which can be divided roughly in two categories:
\begin{itemize}
    \item \textbf{elaboration} is concerned with the \emph{frontend} of proof assistants: how to
    turn requirements expressed in a language as close as possible to natural language and
    standard mathematical notation into statements with precise and unambiguous semantics
    understandable by a computer;
    \item \textbf{synthesis} is closer to the \emph{backend}: the goal is to generate automatically
    parts of or even entire proofs of user-provided statements, including intermediate lemmas
    that may be needed along the way. In the setting of program verification, this can involve
    synthesizing programs themselves, in addition to the proofs that they satisfy their
    specifications. The generated arguments and computations are usually very low-level, and thus
    hidden from the user.
\end{itemize}
Great progress has been made on both fronts in the past decades, and new advances are expected with
the help of state-of-the-art (SOTA) machine learning techniques, including LLMs\footcite{Cite again
Lean FRO roadmap?}.
More theoretical research has also been pursued in the field of
\textbf{type theory}, which studies the logical formalisms at the foundation of all modern ITPs. In
particular, the influential program of \emph{homotopy type theory} kickstarted by Fields medalist
Vladimir Voevodsky has exhibited new ways to understand and mechanize the concept of
\emph{equality}, which is known as an important weakness of traditional type theories.
Type-theoretical foundations are the ultimate backbone on which relies our \emph{trust} in the
output of ITPs, and thus a key differentiator with respect to purely probabilistic approaches to
(generative) AI.

However, little attention has been devoted by the ITP community to another discipline closely
related to type theory, despite its historical primacy on the latter: \textbf{proof theory}. In
particular, \emph{structural} proof theory is its branch concerned with the study of novel
combinatorial structures for representing and manipulating formal proofs. One can identify mainly three motivations for this study:
\begin{enumerate}
    \item the most fundamental is the problem of \textbf{proof identity}, also known as Hilbert's
    24\textsuperscript{th} problem\footcite{strasburger-problem-2019}. It aims to answer the
    philosophical question ``what is a proof?'', and the mathematical question ``when are two proofs
    equal?''. It is thus intimately related to Voevodsky's program in spirit, although we are
    surprisingly \emph{not} aware of any intersections between these two lines of work.

    \item the second motivation is to find proof systems for \textbf{non-standard logics} --- such
    as modal, intermediate, substructural and fixpoint logics --- satisfying good enough
    computational properties as to render an algorithmic treatment of these logics tractable. The
    most important property in this respect is that of \textbf{cut elimination}, which is essential
    both to reduce the complexity of \emph{proof search} (proof \emph{synthesis} in ITP
    terminology), and to ensure productivity of \emph{program execution} through the
    \textbf{Curry-Howard correspondence} (CHC) between proofs and programs. Virtually all these
    logics have found or have been motivated by applications to computer science, and in particular
    program verification. Researchers are also increasingly interested in type-theoretic
    formulations of these logics as they provide expressive languages for specifying behaviors of
    programs that go beyond pure functions, including effects (modal
    logics)\footcite{tangModalEffectTypes2025}, resource-sensitivity (substructural
    logics)\footcite{marshallLinearityUniquenessEntente2022} and recursion (modal/fixpoint
    logics)\footcite{cloustonGuardedLambdaCalculusProgramming2017}.
    
    \item the last motivation is to improve the \textbf{efficiency} of computational procedures on
    proofs. A well-established principle in computer science and software engineering is that
    choosing appropriate data structures for a given problem can lead to orders-of-magnitude
    improvements in algorithmic complexity. While traditional formalisms like \emph{sequent calculi}
    and \emph{tableaux} are recognized for their suitability to proof search with many existing
    implementations, research on the time and space complexity of cut elimination (or more generally
    \emph{proof normalization}) has remained more theoretical. Advances on this question could
    however lead to significant performance gains in synthesis procedures for ITPs, and more
    generally in program execution through the CHC.
\end{enumerate}

In the past decades, two families of proof formalisms have emerged to tackle these challenges:
\begin{itemize}
    \item \textbf{graphical proof systems} represent proof objects as \emph{graphs} instead of the
    traditional \emph{trees} of inference rules. Many programming systems incorporate graph
    representations of programs in their compilation pipeline in order to optimize the flow of
    computation (usually through \emph{graph rewriting} methods), including popular machine learning
    frameworks based on neural networks such as TensorFlow and PyTorch. Indeed, graphs appear
    naturally when one wants to abstract as much as possible from the specific order in which
    operations are performed (e.g. to leverage \emph{parallelism}), or to maximize \emph{sharing} of
    data and computation. \textbf{Proof nets}\footcite{girard-linear-1987} are one of the first
    graphical proof formalisms, initially intended as a way to reduce the bureauceacy of proofs in
    linear logic (Challenge 2) in order to identify their essence (Challenge 1), but also motivated
    by applications to parallel computation (Challenge 3). Further developments stemming from proof
    nets like the \emph{geometry of interaction} and \emph{interaction nets} have since confirmed
    the pertinence of this approach in applications to program optimization\footcite{some reference
    to the works of Dan Ghica and Victor Taelin}. The \emph{combinatorial proofs} of
    Hughes\footcite{Hughes_2006} are also direct decendants of proof nets mainly motivated by the
    problem of proof identity, with a more elaborate graph-theoretic structure involving \emph{skew
    fibrations}. Lastly, there has been increasing interest in \emph{string diagrams} from the
    adjacent field of category theory. String diagrams enjoy nice visual presentations that are also
    amenable to graph-theoretic formulations, have applications both on the
    logical\footcite{bonchi_diagrammatic_2024} and
    computational\footcite{ghicaStringDiagrams$l$calculi2024} sides of the CHC, and have also been
    related to proof nets\footcite{melliesFunctorialBoxesString2006}.

    \item \textbf{deep inference} generalizes Gentzen-style proof systems by allowing inference
    rules to be applied at any depth inside of a formula, rather than being restricted to its
    top-level logical connective\footcite{tubella:hal-02390267}. The terminology of ``deep
    inference'' was proposed by Alessio Guglielmi, who invented the \emph{calculus of structures} to
    overcome the inability of sequent calculus to capture the substructural logic
    $\mathsf{BV}$\footcite{Guglielmi1999ACO}. Since then, calculi of structures and so called
    \emph{nested sequent calculi} have been introduced to give proof systems enjoying
    cut-elimination to most modal\footcite{kuznets_maehara-style_2019} and
    intermediate\footcite{postniece_proof_2010} logics, contributing further to Challenge 2. Deep
    inference has also been used in the study of proof complexity (Challenge 3), providing in some
    cases exponential speedup over sequent calculus with respect to proof
    size\footcite{dasRelativeProofComplexity2015a}, as well as quasipolynomial-time
    cut-elimination\footcite{bruscoliQuasipolynomialNormalisationDeep2016a}. Some formalisms like
    open deduction enjoy CHC-style computational interpretations that improve space
    efficiency\footcite{gundersenAtomicLambdaCalculus2013} (Challenge 3), and deep inference has
    indirectly contributed to Challenge 1 by being an important source of inspiration for
    combinatorial proofs.
\end{itemize}

The experienced researcher (ER) has accumulated significant working knowledge of graphical proof
systems, deep inference proof systems, and combinations thereof, as well as their applications to
ITPs. This expertise was developed during his PhD thesis\footcite{donatoDeepInferenceGraphical2024}
titled ``Deep Inference for Graphical Theorem Proving'', where he designed various proof formalisms
that enable a novel approach to interactive theorem proving based on \textbf{direct manipulation} of
logical statements in a graphical user interface (GUI). This extends earlier works on
\emph{Proof-by-Pointing}\footcite{PbP} and \emph{Proof-by-Linking}\footcite{Chaudhuri2013} --- where
proofs are respectively constructed through \emph{click} and \emph{drag-and-drop} gestures on
formulas --- to a more encompassing paradigm termed \emph{Proof-by-Action} (PbA). The goal was to
improve accessibility and usability of ITPs by focusing on better principles for \emph{human
interaction}, complementing more mainstream research around machine automation. Applying a mixture
of graphical and deep inference proof theory to that effect is a highly original endeavor, with no
similar efforts in the contemporary research landscape.

In continuation of this programme, the ER has introduced in his last
preprint\footcite{donatoScrollNets2025} a new graphical framework called \textbf{scroll nets}. It is
based on a long forgotten diagrammatic proof formalism called \emph{existential graphs} (EGs),
invented by the famous philosopher and logician Charles Sanders Peirce at the dusk of the
19\textsuperscript{th} century --- thus predating the very existence of proof theory and computer
science. Proofs in EGs are defined by a small set of inference rules that \emph{dynamically} rewrite
diagrams in contexts of arbitrary depth, thus combining features of both deep inference and string
diagrams (as already noted by other authors\footcite{bonchi_diagrammatic_2024}). Scroll nets provide
a \emph{static} way to represent proofs in EGs by recording explicitly applications of inference
rules in a \emph{directed graph}, in a way quite reminiscent of proof nets and combinatorial proofs.
Crucially, this directed graph shares the same nodes as the (tree-shaped) statements involved in the
proof, making scroll nets a more compact representation than other graphical proof formalisms, but
also surprisingly a variant of the notion of \textbf{bigraph}. Bigraphs were introduced by Milner as
a foundational combinatorial structure encompassing most models of concurrent/parallel computation,
including Petri nets and his own CCS and
$\pi$-calculus\footcite{milnerBigraphicalReactiveSystems2001}. In his preprint, the ER shows how
scroll nets naturally subsume the simply-typed $\lambda$-calculus --- the common kernel of all type
theories used in ITPs --- by identifying a deep generalization of the notion of \emph{detour} in
natural deduction arising from the inference rules of EGs.

\subsubsection{Specific objectives}

All these discoveries hint toward the potential of scroll nets as a very expressive framework for
both proof theory and type theory, unifying features found in most of the formalisms that have
emerged from the two disciplines over the past decades. While this could be of interest in and of
itself to shape the foundations of a new generation of ITPs, the ER's vision is to exploit the
graphical nature of scroll nets to redesign not only the \emph{backend} but also the \emph{frontend}
of ITPs, making them the \emph{interaction
substrate}\footcite{mackayInteractionSubstratesCombining2025} of a new kind of GUI in the PbA
paradigm.

To realize this vision, the 

% One goal is to make automation (in particular synthesis) easier to understand
% for humans, so as to tighten the interaction loop and thus enhance the collaboration between the
% user and the system. Having more readable inputs and outputs is also crucial to enable \emph(audits)
% of formal developments by external authorities, for instance in the certification of safety-critical
% software by governmental agencies.

Key objective of the project:

\setlength{\fboxsep}{5pt}
\fullwidthbox{\textit{
    Extend the theory of \textbf{scroll nets} to account for richer logics and programming constructs.
}}

Long-term goal of the project:

\setlength{\fboxsep}{5pt}
\fullwidthbox{\textit{
Establish \textbf{scroll nets} as a unifying framework for \textbf{proof theory} and \textbf{type
theory} that supports \textbf{interactive refinement} of formal specifications through
\textbf{diagrammatic manipulations}.
}}

% At a minimum, address the following aspects:

% \begin{itemize}
%     \item Describe the quality and pertinence of the R\&I objectives; are the
%     objectives measurable and verifiable? Are they realistically achievable?

%     \item Describe how your project goes beyond the state-of-the-art, and the
%     extent to which the proposed work is ambitious.
% \end{itemize}

\subsection{Soundness of the proposed methodology
    (including interdisciplinary approaches, consideration of the gender
    dimension and other diversity aspects if relevant for the research project,
    and the quality of open science practices)}
\label{ssc:excellence:methodology}

At a minimum, address the following aspects:

\begin{itemize}
    \item \emph{Overall methodology}: Describe and explain the overall methodology,
    including the concepts, models and assumptions that underpin your work.
    Explain how this will enable you to deliver your project’s objectives. Refer
    to any important challenges you may have identified in the chosen methodology
    and how you intend to overcome them.

    \item \emph{Integration of methods and disciplines to pursue the objectives}:
    Explain how expertise and methods from different disciplines will be brought
    together and integrated in pursuit of your objectives. If you consider that
    an inter-disciplinary\footnotemark{} approach is unnecessary in the
    context of the proposed work, please provide a justification.

    \footnotetext{\emph{Interdisciplinarity} means the integration of information,
    data, techniques, tools, perspectives, concepts or theories from two or
    more scientific disciplines.}

    \item \emph{Gender dimension and other diversity aspects}: Describe how the
    gender dimension and other diversity aspects are taken into account in the
    project's research and innovation content. If you do not consider such a
    gender dimension to be relevant in your project, please provide a
    justification.
    \begin{itemize}
        \item Remember that that this question relates to the \emph{content} of the
        planned research and innovation activities, and not to gender balance
        in the teams in charge of carrying out the project.

        \item  Sex, gender and diversity analysis refers to biological
        characteristics and social/cultural factors respectively. For guidance on
        methods of sex / gender analysis and the issues to be taken into account,
        please refer to this
        \href{https://op.europa.eu/en/publication-detail/-/publication/33b4c99f-2e66-11eb-b27b-01aa75ed71a1/language-en}{link}.
    \end{itemize}

    \item \emph{Open science practices}: Describe how appropriate open science
    practices are implemented as an integral part of the proposed methodology.
    Show how the choice of practices and their implementation is adapted to the
    nature of your work in a way that will increase the chances of the project
    delivering on its objectives \emph{[1/2 page]}. If you believe that none of
    these practices are appropriate for your project, please provide a justification
    here.

    \emph{Open science is an approach based on open cooperative work and systematic
    sharing of knowledge and tools as early and widely as possible in the process.
    Open science practices include early and open sharing of research (for example
    through pre-registration, registered reports, pre-prints, or crowd-sourcing);
    research output management; measures to ensure reproducibility of research
    outputs; providing open access to research outputs (such as publications,
    data, software, models, algorithms, and workflows); participation in open
    peer-review; and involving all relevant knowledge actors including citizens,
    civil society and end users in the co-creation of R\&I agendas and contents
    (such as citizen science).}

    \emph{Please note that this does not refer to outreach actions that may be
    planned as part of the communication, dissemination and exploitation activities.
    These aspects should instead be described below under ``Impact''.}

    \item Proposals selected for funding under Horizon Europe will need to develop
    a detailed data management plan (DMP) for making their data/research outputs
    findable, accessible, interoperable and reusable (FAIR) as a deliverable by
    month 6 and revised towards the end of a project's lifetime. The DMP should
    describe how research outputs (especially research data) generated and/or
    collected during the project will be managed so as to ensure that they are
    findable, accessible, interoperable and reusable.

    \emph{For guidance on open science practices and research data management,
    please refer to the relevant section of the
    \href{https://ec.europa.eu/info/funding-tenders/opportunities/docs/2021-2027/horizon/temp-form/af/af_he-msca-pf_en.pdf}{HE Programme Guide}
    on the Funding \& Tenders Portal.}
\end{itemize}

\subsection{Quality of the supervision, training and of the two-way transfer of
    knowledge between the researcher and the host}
\label{ssc:excellence:supervision}

At a minimum, address the following aspects:

\begin{itemize}
    \item Describe the qualifications and experience of the supervisor(s).
    Provide information regarding the supervisors' level of experience on the
    research topic proposed and their track record of work, including main
    international collaborations, as well as the level of experience in
    supervising/training, especially at advanced level (i.e. PhD and postdoctoral
    researchers).

    \item Planned training activities for the researcher (scientific aspects,
    management/organisation, horizontal and key transferrable skills...).

    \item For \emph{European Fellowships}: two-way transfer of knowledge between
    the researcher and host organisation.

    \item For \emph{Global Fellowships}: three-way transfer of knowledge between
    the researcher, host organisation, and associated partner for outgoing phase.

    \item Rationale and added-value of the non-academic placement (if applicable)
    and secondment (if applicable).
\end{itemize}

Employers and/or funders should ensure that a person is clearly identified to
whom researchers can refer for the performance of their professional duties, and
should inform the researchers accordingly.

Such arrangements should clearly define that the proposed supervisors are
sufficiently expert in supervising research, have the time, knowledge, experience,
expertise and commitment to be able to offer the postdoctoral researcher
appropriate support and provide for the necessary progress and review procedures,
as well as the necessary feedback mechanisms.

\textbf{Supervision} is one of the crucial elements of successful research. Guiding,
supporting, directing, advising and mentoring are key factors for a researcher
to pursue his/her career path. In this context, all MSCA-funded projects are
encouraged to follow the recommendations outlined in the
\href{https://data.europa.eu/doi/10.2766/508311}{MSCA Guidelines on Supervision}
\footnotemark{}.

\footnotetext{While the MSCA Guidelines on Supervision are non-binding,
funded-projects are strongly encouraged to take them into account.}

\subsection{Quality and appropriateness of the researcher's professional
    experience, competences and skills}
\label{ssc:excellence:researcher}

Discuss the quality and appropriateness of the researcher’s existing professional
experience in relation to the proposed research project.

\section{Impact \mscatag{IMP-ACT-IA}}
\label{sc:impact}

\subsection{Credibility of the measures to enhance the career perspectives
    and employability of the researcher and contribution to his/her skills
    development}
\label{ssc:impact:career}

At a minimum, address the following aspects:

\begin{itemize}
    \item Specific measures to enhance career perspectives and employability of
    the researcher inside and/or outside academia.
    \item \emph{Expected} contribution of proposed skills development to the
    future career of the researcher.
\end{itemize}

\subsection{Suitability and quality of the measures to maximise expected
    outcomes and impacts, as set out in the dissemination and exploitation plan,
    including communication activities \mscatag{COM-DIS-VIS-CDV}}
\label{ssc:impact:outcomes}

At a minimum, address the following aspects:

\begin{itemize}
    \item \emph{Plan for the dissemination and exploitation activities, including
    communication activities}\footnotemark{}: Describe the planned measures to
    maximize the impact of your project by providing a first version of your ‘plan
    for the dissemination and exploitation including communication activities’.
    Describe the dissemination, exploitation measures that are planned, and the target
    group(s) addressed (e.g. scientific community, end users, financial actors,
    public at large). Regarding communication measures and public engagement
    strategy, the aim is to inform and reach out to society and show the activities
    performed, and the use and the benefits the project will have for citizens.
    Activities must be strategically planned, with clear objectives, start at the
    outset and continue through the lifetime of the project. The description of
    the communication activities needs to state the main messages as well as the
    tools and channels that will be used to reach out to each of the chosen target
    groups.

    \footnotetext{In case your proposal is selected for funding, a more detailed
    Dissemination and Exploitation plan will need to be provided as a mandatory
    project deliverable during project implementation.}

    \item \emph{Strategy for the management of intellectual property, foreseen
    protection measures}: if relevant, discuss the strategy for the management
    of intellectual property, foreseen protection measures, such as patents,
    design rights, copyright, trade secrets, etc., and how these would be used
    to support exploitation.

    \item All measures should be proportionate to the scale of the project, and
    should contain concrete actions to be implemented both during and after the
    end of the project.
\end{itemize}

\subsection{The magnitude and importance of the project’s contribution to the
    expected scientific, societal and economic impacts}
\label{ssc:impact:future}

\begin{itemize}
    \item Provide a narrative explaining how the project’s results are expected
    to make a difference in terms of impact, beyond the immediate scope and
    duration of the project. The narrative should include the components below,
    tailored to your project.

    \item Be specific, referring to the effects of your project, and not R\&I
    in general in this field. State the target groups that would benefit.

    \item The impacts of your project may be:

    \begin{itemize}
        \item \emph{Scientific}: e.g. contributing to specific scientific advances,
        across and within disciplines, creating new knowledge, reinforcing
        scientific equipment and instruments, computing systems (i.e. research
        infrastructures);

        \item \emph{Economic/technological}: e.g. bringing new
        products, services, business processes to the market, increasing efficiency,
        decreasing costs, increasing profits, contributing to standards’ setting,
        etc.

        \item \emph{Societal}: e.g. decreasing CO2 emissions,
        decreasing avoidable mortality, improving policies and decision-making,
        raising consumer awareness.
    \end{itemize}

    \item Only include such outcomes and impacts where your project would make
    a significant and direct contribution. Avoid describing very tenuous links
    to wider impacts.

    \item Give an indication of the magnitude and importance of the project's
    contribution to the expected outcomes and impacts, should the project be
    successful. Provide credible quantified estimates where possible and meaningful.

    ``Magnitude'' refers to how widespread the outcomes and impacts are likely
    to be. For example, in terms of the size of the target group, or the
    proportion of that group, that should benefit over time.

    ``Importance'' refers to the value of those benefits. For example, number of
    additional healthy life years; efficiency savings in energy supply.
\end{itemize}

\mscatagend{COM-DIS-VIS-CDV} \mscatagend{IMP-ACT-IA}

\section{Quality and Efficiency of the Implementation
         \mscatag{QUA-LIT-QL} \mscatag{WRK-PLA-WP} \mscatag{CON-SOR-CS} \mscatag{PRJ-MGT-PM}}
\label{sc:implementation}

\subsection{Quality and effectiveness of the work plan, assessment of risks and
    appropriateness of the effort assigned to work packages}
\label{ssc:implementation:workplan}

At a minimum, address the following aspects:

\begin{itemize}
    \item Brief presentation of the overall structure of the work plan,
    including deliverables and milestones.

    \item Timing of the different work packages and their components;

    \item Mechanisms in place to assess and mitigate risks (of research and/or
    administrative nature).
\end{itemize}

A Gantt chart must be included and should indicate the proposed Work Packages
(WP), major deliverables, milestones, secondments, placements, if applicable.
This Gantt chart counts towards the 10-page limit.

The schedule in the Gantt chart should indicate the number of months elapsed
from the start of the action (Month 1, Month 2, etc.), but no actual dates.

\subsection{Quality and capacity of the host institutions and participating
organisations, including hosting arrangements}
\label{ssc:implementation:host}

At a minimum, address the following aspects:

\begin{itemize}
    \item Hosting arrangements, including integration in the team/institution(s)
    and support services available to the researcher.

    \item Quality and capacity of the participating organisations, including
    infrastructure, logistics and facilities. Additional information should be
    outlined in Part B-2 Section 5 (``Capacity of the Participating Organisations'').
\end{itemize}

Note that for GF, both the quality and capacity of the outgoing Third Country
host and the return host should be outlined.

\subsection*{Associated partners linked to a beneficiary {\protect\footnotemark{}}}
\label{ssc:implementation:partners}

If applicable, outline here the involvement of any 'associated partners linked
to a beneficiary' (in particular, the name of the entity, the type of link with
the beneficiary and the tasks to be carried out).

\footnotetext{See the definitions section of the MSCA Work Programme for
further information.}

\mscatagend{CON-SOR-CS} \mscatagend{PRJ-MGT-PM} \mscatagend{QUA-LIT-QL} \mscatagend{WRK-PLA-WP}
\end{document}

% kate: default-dictionary en_GB;
