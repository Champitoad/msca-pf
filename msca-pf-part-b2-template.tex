\documentclass[11pt,draftproposal]{msca-pf}
% {{{ packages

% better environments
\usepackage[nameinlink, noabbrev]{cleveref}
\usepackage[shortlabels]{enumitem}
\usepackage{booktabs}
\usepackage{caption}
\usepackage{subcaption}

% better typography
\usepackage[activate={true,nocompatibility}, % activate protrusion and font expansion
            final,              % enable microtype, use draft to disable
            tracking=true,
            factor=1100,        % more protrusion
            stretch=10,         % smaller values (default 20, 20) to avoid blurring
            shrink=10]{microtype}
\SetTracking{encoding={*}, shape=sc}{40}

% }}}

% {{{ formatting

% NOTE: this needs to be kept so that the sections match the template!
\setcounter{section}{3}

% }}}

% {{{ commands

\NewDocumentCommand \ra {} {\raggedright\arraybackslash}


% }}}

% {{{ title

\title{Part B-2}
\author{}
\date{}

% NOTE: update these as necessary
% \mscaidentifier{HORIZON-MSCA-2024-PF-01}
% \mscaproject{TDB}

% }}}

\begin{document}

\maketitle

\textit{The text in each section is provided only as guidance and should be deleted
in the final application (including this text). However, some of the tables
are part of the document format and should be used and expanded from the ones
provided here.}

\textit{When in doubt, always consult the official template and programme guide!}

\section{CV of the researcher (indicative length: 5 pages)}

Any information provided in Parts A and B of the proposal should be fully consistent.
Always mention full dates (using format: dd/mm/yyyy). The CV should include the
standard academic and research record. Any research career gaps and/or unconventional
paths should be clearly explained.

At a minimum, the CV should contain:

\begin{enumerate}[a)]
    \item The name of the researcher;
    \item Professional experience (most recent first, with exact dates in format
    dd/mm/yyyy);
    \item Education, including PhD award date (most recent first, with exact
    dates in format: dd/mm/yyyy).
\end{enumerate}

The CV should include information on:

\begin{itemize}
    \item Publications in peer-reviewed scientific journals, peer-reviewed conference
    proceedings,  and/or monographs (they are expected to be open access either
    published or through repositories) and other outputs such as data, software,
    algorithms significant for your research path (they are expected to be open
    access in appropriate repositories to the extent possible; they should be
    accompanied by a very short qualitative assessment of their scientific
    significance and not by the Journal Impact Factor);

    \item Invited presentations to internationally established conferences and/or
    international advanced schools;

    \item Organisation of international conferences, including membership in
    the steering and/or programme committee;

    \item Research expeditions led by the researcher;

    \item Granted patent(s);

    \item Examples of participation in industrial innovation;

    \item Prizes and Awards;

    \item Funding received so far;

    \item Supervising and mentoring activities;

    \item Other items of interest.
\end{itemize}

Applicants who have successfully defended their doctoral thesis \emph{before} the call
deadline but who have not yet formally been awarded the doctoral degree must
clearly indicate the date of the successful PhD defence (``viva''). Researchers
having their last thesis defence after the call deadline will be automatically
declared ineligible for this call.

\subsection{Template CV}

This template is only added to showcase the commands used to create the CV. As
stated above, there is no mandatory format and these commands are mainly added
as inspiration.

\subsubsection*{Education}

\begin{cvitem}
\cventry{DD/MM/YYYY DD/MM/YYYY}{Ph.D. in Aerospace Engineering}{University Name}{Location}
\cvdetail{Title}{Title of my Ph.D}
\cvdetail{Advisor}{John Doe}
\end{cvitem}

\subsubsection*{Work Experience}

\begin{cvitem}
\cventry{DD/MM/YYYY DD/MM/YYYY}{Job Title}{Company / University}{Location}
\cvdetail{Description}{Job description in short}
\end{cvitem}

\subsubsection*{Publications}

\begin{cvitem}
\cvpub{DD/MM/YYYY}{John Doe, Jane Doe}{Title of Paper}{Journal Name, Vol. XX, pp. XX--XX}
\cvdetail{Description}{Main findings of the paper}
\cvdetail{URL}{DOI or arXiV URL}
\end{cvitem}

\section{Capacity of the Participating Organisation(s)}

Please provide an overview list of all participating organisations (the
beneficiary and, where applicable, all associated partners) using template
table~\ref{ssc:table1} below, and more detailed information for each of the
participating organisations (using a separate table for each organisation) using
template table~\ref{ssc:table2} below.

Any inter-relationship between the participating organisation(s) or individuals
and other entities/persons appearing (e.g. family ties, shared premises or
facilities, joint ownership, financial interest, overlapping staff or directors,
etc.) must be declared in the proposal.

Applicants should provide additional information regarding the administrative/legal
relations between the department carrying out the work as described in the table
below, and the entity/entities mentioned in Part A of the proposal (i.e. linked
to the given Participant Identification Code –- PIC).

Should the proposal be shortlisted for funding, all participating organisations
will have to be registered with the European Commission’s
\href{https://ec.europa.eu/info/funding-tenders/opportunities/portal/screen/how-to-participate/participant-register}{Participant Register Services}. Therefore where
this information is \href{https://ec.europa.eu/info/funding-tenders/opportunities/portal/screen/how-to-participate/participant-register-search}{already known},
please provide in Table 5.1 the (draft or validated) nine digit \emph{Participant
Identification Code} (PIC) for the beneficiary and, where applicable, each
associated partner.

\subsection{Template table: Overview of Participating Organisations}
\label{ssc:table1}

Only relevant rows from the table should be kept.

\begin{mscaorgoverview}
Beneficiary &
&
&
&
& \\
\hline
Associated partner linked to a beneficiary (if applicable) &
&
&
&
& \\
\hline
Associated partner for outgoing phase (mandatory for GF) &
&
&
&
& \\
\hline
Associated partner for secondment (optional) &
&
&
&
& \\
\hline
Associated partner for non-academic placement (optional) &
&
&
&
& \\
\hline
Other &
&
&
&
& \\
\hline
\end{mscaorgoverview}

\subsection{Template table: Capacity of the Participating Organisations}
\label{ssc:table2}

Please complete a separate table for each participating organisation. For the
beneficiary, this table should be \emph{maximum 1 page in length}; for each associated
partner, the table should be \emph{maximum ½ page in length}. Choose one of

\begin{itemize}
    \item Beneficiary (compulsory).
    \item Associated partner linked to a beneficiary (if applicable)
    \item Associated partner for outgoing phase (compulsory for GF only)
    \item Associated partner for secondment (optional)
    \item Associated partner for non-academic placement (optional)
\end{itemize}

\begin{mscaorgcapacity}
\multicolumn{2}{|l|}{
    \textbf{[Role]}
} \\
\hline
\multicolumn{2}{|l|}{
    \textbf{[Full name + Legal Entity Short Name + Country]}
} \\
\hline
\multicolumn{2}{|l|}{
    \textbf{General description}
} \\
\hline
\textbf{Role and profile of supervisor} & \\
\hline
\textbf{Key research facilities, infrastructure and equipment} &
\emph{Demonstrate that the beneficiary has sufficient facilities and infrastructure
to host and/or offer a suitable environment for training and transfer of knowledge
to the recruited experienced researcher.}

\bigskip

\emph{If applicable, indicate the name of the associated partner linked to a beneficiary
and describe the nature of the link in the corresponding table.} \\
\hline
\textbf{Previous and current involvement in EU-funded research and training
programmes/actions/projects} &
\emph{Indicate up to 5 relevant EU, national or international research and training
actions/projects in which the institution/department has previously participated
and/or is currently participating.} \\
\hline
\end{mscaorgcapacity}

\section{Additional ethics information}

Additional information that could not be included in Part A of the proposal (if needed).

\section{Additional information on security screening}

Additional information on security aspects that could not be included in Part
A of the proposal (if needed).

\section{Environmental considerations in light of MSCA Green Charter}

Please explain how the proposed project would strive to adhere to the MSCA Green
Charter\footnotemark{} during its implementation. Please indicate here --
max 1/2 page -- what actions you propose to take to ensure the sustainable
implementation of project and to mitigate its environmental impact, in line
with the principles set out in the MSCA Green Charter.

\footnotetext{MSCA Green Charter \url{https://ec.europa.eu/msca/green_charter}.}

\section{
    Required for Global Fellowships only: Letter(s) of commitment from
    associated partners (hosting of the outgoing phase)}

Use this section to add scanned copies of the letter(s) of commitment, if applicable.

Minimum requirements:

\begin{itemize}
    \item With heading or stamp from the institution;
    \item Up-to-date document, i.e. not dated prior to the call publication;
    \item Demonstrating the will to actively participate in the (identified) proposal;
    \item Explanation of the precise role.
\end{itemize}

Any additional information the organisation deems useful can be added in the letter.

Note that the expert evaluators will be instructed to disregard the contribution
of any associated partners for which no such evidence of commitment is submitted.

In case the letter fails to provide enough information on the associated partner's
role and/or enough assurance of their commitment in the project (e.g. no signature,
wrong proposal references, outdated letter…), the experts may penalise the
proposal on these aspects under the implementation evaluation criterion.

For GF proposals \emph{the absence of a letter of commitment will render the
proposal inadmissible and the proposal will not be evaluated}.

\subsection*{Non-binding example of template letter of commitment for PF associated partners}

I undersigned \emph{[title, first name and surname]}, in my quality of
\emph{[role in the organisation]} in \emph{[name of the organisation]} commit
to set up all necessary provisions to participate as associated partner in the
proposal \emph{[proposal number and/or acronym]} submitted to the call
HE-MSCA-2024-PF, should the proposal be funded.

On behalf of \emph{[name of the organisation]}, I also confirm that we will
participate and contribute to the research, innovation and training activities
as planned in this project. In particular, \emph{[name of the organisation]}
will be involved in \emph{[free field for any additional information that the
participating organisation wishes to indicate in order to describe its role and
contribution to the project]}.

I hereby declare that I am entitled to commit into this process the entity I represent.

\hfill \emph{Name, Date, Signature}

\end{document}

% kate: default-dictionary en_GB;
