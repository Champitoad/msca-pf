\documentclass[11pt,draftproposal]{msca-pf}
% {{{ packages

% better environments
\usepackage[nameinlink, noabbrev]{cleveref}
\usepackage[shortlabels]{enumitem}
\usepackage{booktabs}
\usepackage{caption}
\usepackage{subcaption}

% better typography
\usepackage[activate={true,nocompatibility}, % activate protrusion and font expansion
            final,              % enable microtype, use draft to disable
            tracking=true,
            factor=1100,        % more protrusion
            stretch=10,         % smaller values (default 20, 20) to avoid blurring
            shrink=10]{microtype}
\SetTracking{encoding={*}, shape=sc}{40}

% }}}

% {{{ formatting

% NOTE: this needs to be kept so that the sections match the template!
\setcounter{section}{3}

% }}}

% {{{ commands

% add your own fancy commands
% \NewDocumentCommand \mycmd { m } {\textbf{#1}}

\newcommand{\proj}{\small\textsc{ScrollNets}}

% }}}

% {{{ title

\title{Part B-2}
\author{}
\date{}

% NOTE: update these as necessary
% \mscaidentifier{HORIZON-MSCA-2025-PF-01}
% \mscaproject{TDB}

% }}}

\begin{document}

\maketitle

\section{CV of the researcher (indicative length: 5 pages)}
\label{sc:cv}

Any information provided in Parts A and B of the proposal should be fully consistent.
Always mention full consecutive dates (using format: dd/mm/yyyy). The CV should
include the standard academic and research record. Any research career gaps
and/or unconventional paths should be clearly explained and the dates must match
the ones provided in Part A (if applicable).

At a minimum, the CV should contain:

\begin{enumerate}[a)]
    \item The name of the researcher;
    \item Professional experience (most recent first, with exact dates in format
    dd/mm/yyyy);
    \item Education, including PhD award date (most recent first, with exact
    dates in format: dd/mm/yyyy).
\end{enumerate}

The CV should include information on:

\begin{itemize}
    \item Publications in peer-reviewed scientific journals, peer-reviewed conference
    proceedings,  and/or monographs (they are expected to be open access either
    published or through repositories) and other outputs such as data, software,
    algorithms significant for your research path (they are expected to be open
    access in appropriate repositories to the extent possible; they should be
    accompanied by a very short qualitative assessment of their scientific
    significance and not by the Journal Impact Factor);

    \item Invited presentations to internationally established conferences and/or
    international advanced schools;

    \item Organisation of international conferences, including membership in
    the steering and/or programme committee;

    \item Research expeditions led by the researcher;

    \item Granted patent(s);

    \item Examples of participation in industrial innovation;

    \item Prizes and Awards;

    \item Funding received so far;

    \item Supervising and mentoring activities;

    \item Other items of interest.
\end{itemize}

Applicants who have successfully defended their doctoral thesis \emph{before} the call
deadline but who have not yet formally been awarded the doctoral degree must
clearly indicate the date of the successful PhD defence (``viva''). Researchers
having their last thesis defence after the call deadline will be automatically
declared ineligible for this call.

\section{Capacity of the Participating Organisation(s)}
\label{sc:organisations}

Please provide an overview list of all participating organisations (the
beneficiary and, where applicable, all associated partners) using template
table~\ref{ssc:organisations:overview} below, and more detailed information for
each of the participating organisations (using a separate table for each
organisation) using template table~\ref{ssc:organisations:capacity} below.

Any inter-relationship between the participating organisation(s) or individuals
and other entities/persons appearing (e.g. family ties, shared premises or
facilities, joint ownership, financial interest, overlapping staff or directors,
etc.) must be declared in the proposal.

Applicants should provide additional information regarding the administrative/legal
relations between the department carrying out the work as described in the table
below, and the entity/entities mentioned in Part A of the proposal (i.e. linked
to the given Participant Identification Code –- PIC).

Should the proposal be shortlisted for funding, all participating organisations
will have to be registered with the European Commission’s
\href{https://ec.europa.eu/info/funding-tenders/opportunities/portal/screen/how-to-participate/participant-register}{Participant Register Services}. Therefore where
this information is \href{https://ec.europa.eu/info/funding-tenders/opportunities/portal/screen/how-to-participate/participant-register-search}{already known},
please provide in Table 5.1 the (draft or validated) nine digit \emph{Participant
Identification Code} (PIC) for the beneficiary and, where applicable, each
associated partner.

\subsection{Template table: Overview of Participating Organisations}
\label{ssc:organisations:overview}

Only relevant rows from the table should be kept.

\begin{mscaorgoverview}
Beneficiary &
&
&
&
& \\
\hline
Associated partner linked to a beneficiary (if applicable) &
&
&
&
& \\
\hline
Associated partner for outgoing phase (mandatory for GF) &
&
&
&
& \\
\hline
Associated partner for secondment (if applicable) &
&
&
&
& \\
\hline
Associated partner for non-academic placement (if applicable) &
&
&
&
& \\
\hline
\end{mscaorgoverview}

\subsection{Template table: Capacity of the Participating Organisations}
\label{ssc:organisations:capacity}

Please complete a separate table for each participating organisation. For the
beneficiary, this table should be \emph{maximum 1 page in length}; for each associated
partner, the table should be \emph{maximum ½ page in length}. Choose one of

\begin{itemize}
    \item Beneficiary (compulsory).
    \item Associated partner linked to a beneficiary (if applicable)
    \item Associated partner for outgoing phase (compulsory for GF only)
    \item Associated partner for secondment (if applicable)
    \item Associated partner for non-academic placement (if applicable)
\end{itemize}

\begin{mscaorgcapacity}
\multicolumn{2}{|l|}{
    \textbf{[Role]}
} \\
\hline
\multicolumn{2}{|l|}{
    \textbf{[Full name + Legal Entity Short Name + Country]}
} \\
\hline
\multicolumn{2}{|l|}{
    \textbf{General description}
} \\
\hline
\textbf{Role and profile of supervisor} & \\
\hline
\textbf{Key research facilities, infrastructure and equipment} &
\emph{Demonstrate that the beneficiary has sufficient facilities and infrastructure
to host and/or offer a suitable environment for training and transfer of knowledge
to the recruited experienced researcher.}

\bigskip

\emph{If applicable, indicate the name of the associated partner linked to a beneficiary
and describe the nature of the link in the corresponding table.} \\
\hline
\textbf{Previous and current involvement in EU-funded research and training
programmes/actions/projects} &
\emph{Indicate up to 5 relevant EU, national or international research and training
actions/projects in which the institution/department has previously participated
and/or is currently participating.} \\
\hline
\end{mscaorgcapacity}

\section{Additional ethics information}
\label{sc:ethics}

All relevant ethics information has been provided in Part A of the proposal. No further details are necessary.

\section{Additional information on security screening}
\label{sc:security}

The {\proj} project does not involve any security-sensitive research. Therefore, no additional
information on security screening is required.

\section{Environmental considerations in light of MSCA Green Charter}
\label{sc:environment}

The {\proj} project is committed to sustainable research practices in line with the MSCA Green
Charter and the University of Birmingham's sustainability policies. As a primarily theoretical
project without any generative AI components, its environmental footprint is inherently low.
Nevertheless, specific measures will be taken to minimize environmental impact throughout the
project's lifecycle.

All research outputs, including publications and software prototypes, will be disseminated digitally
via open-access journals, pre-print servers like arXiv, and the project website, eliminating the
need for printed materials. For collaboration and dissemination events, virtual participation will
be prioritized. When travel is essential for conferences or meetings, low-carbon transport options,
such as rail, will be chosen whenever feasible. The computational work will be conducted on the
energy-efficient computing infrastructure provided by the University of Birmingham. These actions
ensure that the project not only advances scientific knowledge but does so in a socially and
environmentally responsible manner.

\section{
    Required for Global Fellowships only: Letter(s) of commitment from
    associated partners (hosting of the outgoing phase)}
\label{sc:commitment_letter}

This proposal is submitted for a European Fellowship and, as such, does not require a letter of
commitment for an outgoing phase, as this is not a Global Fellowship.

\section{
    Disclaimer on the use of generative AI for the preparation of the proposal
}

The following generative AI technologies have been used in the preparation of this proposal:

\begin{itemize}
\item Chatbots like OpenAI's ChatGPT and Mistral's Le Chat as search engines, including the use of the Deep Search functionality. No output from these chatbots has been copy-pasted, and proposed sources were rigorously checked manually for scientific integrity before integration as references in the proposal's footnotes.
\item The GitHub Copilot Agent feature of Visual Studio Code loaded with models GPT5 and Gemini 2.5 Pro,
  as a tool for assistance in writing. More specifically, it was prompted for summarization and
  reflowing of sentences/paragraphs. When used for idea suggestion, output has been rigorously
  checked for factuality and alignment with the author's original intent and beliefs.
\end{itemize}

The author claims full responsibility for everything written in this proposal, including
eventual copyright infringements that might have escaped his scrutiny.

\end{document}

% kate: default-dictionary en_GB;
